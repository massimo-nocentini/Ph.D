

\documentclass[11pt]{article}
\usepackage[italian]{babel}
\usepackage{geometry}
\usepackage[nochapters,eulermath,beramono]{classicthesis}
\usepackage{hyperref}
\geometry{
  a4paper, % oppure letter etc
  tmargin=1.5in,
  bmargin=1.5in,
  lmargin=1.5in,
  rmargin=1.5in,
  hscale=0.65,
  vscale=0.75,
  %twosideshift=30pt
  }
%% Paragrafi:
%\def \n {\noindent}
%\baselineskip=18truept





\begin{document}


\begin{center}

\large{\bf{Dottorato di Ricerca in Matematica, Informatica, Statistica\\
Firenze-Perugia-INdAM\\}}
\vskip .5cm
\large{Scheda Personale - Primo anno}
\vskip .5cm


\end{center}


\vskip 1cm


\begin{description}


\item[Cognome:] \emph{Nocentini}

\item[Nome:] \emph{Massimo}

\item[e-mail preferita:]  \href{mailto:massimo.nocentini@gmail.com}{massimo.nocentini@gmail.com}

\item[Ciclo: ] $XXXI$

\item[Anno accademico: ] $2015-2016$

%\item[Tutor:] 

%\item[e-mail:] 

\item[Referente interno al Collegio:] \emph{Donatella Merlini}

\item[e-mail:] \href{mailto:donatella.merlini@unifi.it}{donatella.merlini@unifi.it}

\end{description}



\section*{Piano di studio}

Not planned yet.

\section*{Corsi interessanti}

\begin{enumerate}

\item \href{http://www.unifi.it/index.php?module=ofform2&mode=1&cmd=3&AA=2015&afId=359747}{B013019 - CRITTOGRAFIA},
    \begin{description}
        \item[corso di laurea] Laurea Triennale (DM 270/04) - MATEMATICA
        \item[Settore Scientifico disciplinare] MAT/02 - ALGEBRA
        \item[Periodo didattico] dal 01/03/2016 al 17/06/2016
        \item[CFU] 6
        \item[Ore] 48
        \item[Docente] \href{http://www.unifi.it/p-doc2-2015-0-A-2b333c2b3228-0.html}{Orazio Puglisi}

    \end{description}

\item \href{http://www.unifi.it/index.php?module=ofform2&mode=1&cmd=3&AA=2015&afId=426613}{B018796 - TEORIA DEI GRAFI E COMBINATORIA},
    \begin{description}
        \item[corso di laurea] Laurea Magistrale - MATEMATICA
        \item[Settore Scientifico disciplinare] MAT/02 - ALGEBRA
        \item[Periodo didattico] dal 01/03/2016 al 17/06/2016
        \item[CFU] 9
        \item[Ore] 72
        \item[Docente] \href{http://www.unifi.it/p-doc2-2015-0-A-2b333a29342e-0.html}{Carlo Casolo}

    \end{description}

\item \href{http://www.unifi.it/index.php?module=ofform2&mode=1&cmd=3&AA=2015&afId=427440}{B020988 - TEMI AVANZATI DI LOGICA},
    \begin{description}
        \item[corso di laurea] Corsi di Laurea Magistrale - LOGICA, FILOSOFIA E STORIA DELLA SCIENZA
        \item[Settore Scientifico disciplinare] M-FIL/02 - LOGICA E FILOSOFIA DELLA SCIENZA
        \item[Periodo didattico] dal 01/03/2016 al 01/06/2016
        \item[CFU] 12 
        \item[Ore] 72
        \item[Docente] \href{http://www.unifi.it/p-doc2-2015-0-A-2b333b2c362a-0.html}{Pierluigi Minari}

    \end{description}

\item \href{http://www.unifi.it/index.php?module=ofform2&mode=1&cmd=3&AA=2015&afId=426607}{B018797 - TEORIA DEI NUMERI},
    \begin{description}
        \item[corso di laurea] Corsi di Laurea Magistrale - MATEMATICA
        \item[Settore Scientifico disciplinare] MAT/02 - ALGEBRA
        \item[Periodo didattico] dal 01/03/2016 al 17/06/2016
        \item[CFU] 9 
        \item[Ore] 72
        \item[Docente] \href{http://www.unifi.it/index.php?module=ofform2&mode=1&cmd=3&AA=2015&afId=426607}{Virgilio Pannone}

    \end{description}

\item \href{}{Analysis of algorithms and data structures through Riordan arrays},
    \begin{description}
        \item[corso di laurea] Corso di Dottorato in Matematica, Informatica e Statistica
        %\item[Settore Scientifico disciplinare] MAT/02 - ALGEBRA
        \item[Periodo didattico] gennaio/febbraio 2016
        \item[CFU] 4+2
        \item[Ore] 12
        \item[Docente] \href{http://local.disia.unifi.it/merlini/}{Donatella Merlini}

    \end{description}
\end{enumerate}



\section*{Corsi seguiti}

\begin{description}

\item[%inserire le date
] 
% inserire il titolo del corso

\item[%inserire le date
] 
% inserire il titolo del corso

\end{description}

\section*{Esami sostenuti}

\begin{description}

\item[%inserire le date
] 
% inserire il titolo del corso

\item[%inserire le date
] 
% inserire il titolo del corso

\end{description}

\section*{Scuole estive e conferenze interessanti}

\begin{itemize}
    \item \href{https://www.cs.ox.ac.uk/projects/utgp/school/index.html}
        {Summer School on Generic and Effectful Programming}
    \item \href{http://www.utrechtsummerschool.nl/courses/science/applied-functional-programming-in-haskell}
        {Applied Functional Programming in Haskell}
    \item \href{http://redex.racket-lang.org/summer-school.html}{Redex summer school}
    \item \href{http://www.easyconferences.eu/icalp2016/}{Colloquium on Automata, Languages, and Programming}
    \item \href{http://grammars.grlmc.com/LATA2016/}{Language and Automata Theory and Applications}
    \item \href{http://www.aofa2016.meetings.pl/}{International Conference on Probabilistic, Combinatorial and Asymptotic Methods 
    for the Analysis of Algorithms}
    \item \href{http://conf.researchr.org/home/icfp-2016}{ICFP 2016}
\end{itemize}


\section*{Stato di avanzamento della tesi}

\begin{description}

\item[Argomento:] 

\item[Stato di avanzamento:] 

\end{description}




\section*{Seminari tenuti}

\begin{description}

\item[%inserire la data
] 
% inserire il titolo

\item[%inserire la data
Seminario sull'argomento di tesi] 
% inserire il titolo


\end{description}



\section*{Seminari seguiti}

\begin{description}

\item[%inserire la data
] 
% inserire il titolo

\item[%inserire la data
] 
% inserire il titolo

\end{description}


\section*{Congressi, convegni o soggiorni all'estero}

\begin{description}

\item[%inserire le date
] 
% inserire il titolo della conferenza o la descrizione della visita

\end{description}


\section*{Eventuale attivit\`a didattica}

\begin{description}

\item[%inserire le date
] 
% inserire il titolo del corso

\end{description}

\section*{Altra attivit\`a scientifica}

\begin{description}

\item[%inserire le date
] 
%inserire evento

\end{description}


\section*{Eventuali pubblicazioni o articoli inviati}

\begin{description}


\item[Pubblicazioni] 


\item[Articoli inviati]





\end{description}




\section*{Gruppo di ricerca e fondi ai quali afferisce}

\begin{description}

\item[Gruppo INdAM] %inserire GNAMPA, GNCS, GNFM o GNSAGA

\item[Ex 60\%] Responsabile %inserire nome responsabile

\item[PRIN] Titolo\hskip 2cm Responsabile %inserire nome responsabile; 


\end{description}


\vfill

\end{document}


