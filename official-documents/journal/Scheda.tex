

\documentclass[11pt]{article}
\usepackage[italian]{babel}
\usepackage{geometry}
\usepackage[nochapters,eulermath,beramono]{classicthesis}
\usepackage{hyperref}
\geometry{
  a4paper, % oppure letter etc
  tmargin=1.5in,
  bmargin=1.5in,
  lmargin=1.5in,
  rmargin=1.5in,
  hscale=0.65,
  vscale=0.75,
  %twosideshift=30pt
  }
%% Paragrafi:
%\def \n {\noindent}
%\baselineskip=18truept





\begin{document}


\begin{center}

\large{\bf{Dottorato di Ricerca in Matematica, Informatica, Statistica\\
Firenze-Perugia-INdAM\\}}
\vskip .5cm
\large{Scheda Personale - Primo anno}
\vskip .5cm


\end{center}


\vskip 1cm


\begin{description}


\item[Cognome:] \emph{Nocentini}

\item[Nome:] \emph{Massimo}

\item[e-mail preferita:]  \href{mailto:massimo.nocentini@gmail.com}{massimo.nocentini@gmail.com}

\item[Ciclo: ] $XXXI$

\item[Anno accademico: ] $2015-2016$

%\item[Tutor:] 

%\item[e-mail:] 

\item[Referente interno al Collegio:] \emph{Donatella Merlini}

\item[e-mail:] \href{mailto:donatella.merlini@unifi.it}{donatella.merlini@unifi.it}

\end{description}



\section*{Piano di studio}

\begin{enumerate}

\item \href{http://www.unifi.it/index.php?module=ofform2&mode=1&cmd=3&AA=2015&afId=426613}{B018796 - TEORIA DEI GRAFI E COMBINATORIA},
    \begin{description}
        \item[corso di laurea] Laurea Magistrale - MATEMATICA
        \item[Settore Scientifico disciplinare] MAT/02 - ALGEBRA
        \item[Periodo didattico] dal 01/03/2016 al 17/06/2016
        \item[CFU] 9
        \item[Ore] 72
        \item[Docente] \href{http://www.unifi.it/p-doc2-2015-0-A-2b333a29342e-0.html}{Carlo Casolo}

    \end{description}

\item \href{}{Introduction to Process Calculi and related Type Systems},
    \begin{description}
        \item[corso di laurea] Corsi di Dottorato in Informatica, Universit\`a di Pisa
        %\item[Settore Scientifico disciplinare] MAT/02 - ALGEBRA
        \item[Periodo didattico] December 2015
        \item[CFU] 4 
        \item[Ore] 20
        \item[Docente] \href{http://www-sop.inria.fr/members/Ilaria.Castellani/Home.html}{Ilaria Castellani}, 
            \href{http://local.disia.unifi.it/pugliese/}{Rosario Pugliese} 

    \end{description}

\item \href{}{Analysis of algorithms and data structures through Riordan arrays},
    \begin{description}
        \item[corso di laurea] Corso di Dottorato in Matematica, Informatica e Statistica
        %\item[Settore Scientifico disciplinare] MAT/02 - ALGEBRA
        \item[Periodo didattico] gennaio/febbraio 2016
        \item[CFU] 4+2
        \item[Ore] 12
        \item[Docente] \href{http://local.disia.unifi.it/merlini/}{Donatella Merlini}

    \end{description}
\end{enumerate}

\section*{Corsi interessanti}

\begin{enumerate}

\item \href{http://www.di.unipi.it/en/phd/phd-teaching/phd-courses/2016/891-coalgebre-automi-e-linguaggi-nominali-ecc}{Coalgebre, automi e linguaggi nominali},
    \begin{description}
        \item[corso di laurea] PhD in Computer Science, University of Pisa
        %\item[Settore Scientifico disciplinare] MAT/02 - ALGEBRA
        \item[Periodo didattico] June 2016, 20 to July 2016, 1
        %\item[CFU] 4+2
        %\item[Ore] 12
        \item[Docente] \href{http://www.mimuw.edu.pl/~klin/}{Bartek Klin}

    \end{description}

\item \href{http://www.unifi.it/index.php?module=ofform2&mode=1&cmd=3&AA=2015&afId=359747}{B013019 - CRITTOGRAFIA},
    \begin{description}
        \item[corso di laurea] Laurea Triennale (DM 270/04) - MATEMATICA
        \item[Settore Scientifico disciplinare] MAT/02 - ALGEBRA
        \item[Periodo didattico] dal 01/03/2016 al 17/06/2016
        \item[CFU] 6
        \item[Ore] 48
        \item[Docente] \href{http://www.unifi.it/p-doc2-2015-0-A-2b333c2b3228-0.html}{Orazio Puglisi}

    \end{description}

\item \href{http://www.unifi.it/index.php?module=ofform2&mode=1&cmd=3&AA=2015&afId=427440}{B020988 - TEMI AVANZATI DI LOGICA},
    \begin{description}
        \item[corso di laurea] Corsi di Laurea Magistrale - LOGICA, FILOSOFIA E STORIA DELLA SCIENZA
        \item[Settore Scientifico disciplinare] M-FIL/02 - LOGICA E FILOSOFIA DELLA SCIENZA
        \item[Periodo didattico] dal 01/03/2016 al 01/06/2016
        \item[CFU] 12 
        \item[Ore] 72
        \item[Docente] \href{http://www.unifi.it/p-doc2-2015-0-A-2b333b2c362a-0.html}{Pierluigi Minari}

    \end{description}

\item \href{http://www.unifi.it/index.php?module=ofform2&mode=1&cmd=3&AA=2015&afId=426607}{B018797 - TEORIA DEI NUMERI},
    \begin{description}
        \item[corso di laurea] Corsi di Laurea Magistrale - MATEMATICA
        \item[Settore Scientifico disciplinare] MAT/02 - ALGEBRA
        \item[Periodo didattico] dal 01/03/2016 al 17/06/2016
        \item[CFU] 9 
        \item[Ore] 72
        \item[Docente] \href{http://www.unifi.it/index.php?module=ofform2&mode=1&cmd=3&AA=2015&afId=426607}{Virgilio Pannone}

    \end{description}

\end{enumerate}



\section*{Corsi seguiti}

\begin{description}

\item[January 2016, 18 up to March 2016, 21] Writing for Research, level B2/C1, 
    taught by Dr. \emph{Lisa Annemarie Weber}\footnote{\url{lisa.weber@unifi.it}},
    2 hours/week.
\item[December 2015] Introduction to Process Calculi and related Type Systems

\item[%inserire le date
] 
% inserire il titolo del corso

\end{description}

\section*{Esami sostenuti}

\begin{description}

\item[%inserire le date
] 
% inserire il titolo del corso

\item[%inserire le date
] 
% inserire il titolo del corso

\end{description}

\section*{Scuole estive e conferenze interessanti}

\begin{itemize}
    \item \href{https://www.cs.ox.ac.uk/projects/utgp/school/index.html}
        {Summer School on Generic and Effectful Programming}
    \item \href{http://www.utrechtsummerschool.nl/courses/science/applied-functional-programming-in-haskell}
        {Applied Functional Programming in Haskell}
    \item \href{http://redex.racket-lang.org/summer-school.html}{Redex summer school}
    \item \href{http://www.easyconferences.eu/icalp2016/}{Colloquium on Automata, Languages, and Programming}
    \item \href{http://grammars.grlmc.com/LATA2016/}{Language and Automata Theory and Applications}
    \item \href{http://www.aofa2016.meetings.pl/}{International Conference on Probabilistic, Combinatorial and Asymptotic Methods 
    for the Analysis of Algorithms}
    \item \href{http://conf.researchr.org/home/icfp-2016}{ICFP 2016}
\end{itemize}


\section*{Stato di avanzamento della tesi}

\begin{description}

\item[Argomento:] 

\item[Stato di avanzamento:] 

\end{description}




\section*{Seminari tenuti}

\begin{description}

\item[%inserire la data
] 
% inserire il titolo

\item[%inserire la data
Seminario sull'argomento di tesi] 
% inserire il titolo


\end{description}



\section*{Seminari seguiti}

\begin{description}

    \item[$19/11/2015$] Summary of first year activities given by students enrolled in $XXX$ cycle of PhD
        course in Mathematics, Informatics and Statistics, University of Florence and Perugia.

\end{description}


\section*{Congressi, convegni o soggiorni all'estero}

\begin{description}

\item[%inserire le date
] 
% inserire il titolo della conferenza o la descrizione della visita

\end{description}


\section*{Eventuale attivit\`a didattica}

\begin{description}

\item[26/11/2015: introduction to SymPy, a computer algebra system] a lecture for 
    \emph{B018968 - PROGETTAZIONE E ANALISI DI ALGORITMI}\footnote{\url{http://www.unifi.it/index.php?module=ofform2\&mode=1\&cmd=3\&AA=2015\&afId=426024}} 
    course, 3 hours. Content: an introduction to 
    \emph{SymPy}\footnote{\url{http://www.sympy.org/en/index.html}}, covered object and meta language, basic symbolic 
    manipulations, Taylor series expansions, Fibonacci numbers and solved Quicksort's swap recurrence
    in the average case. We provided an \emph{IPython notebook}, released as an open-source project within
    my \emph{GitHub} repository \footnote{\url{https://github.com/massimo-nocentini/PhD/tree/master/courses/paa}}
    and browseable on line\footnote{\url{http://nbviewer.jupyter.org/github/massimo-nocentini/PhD/blob/master/courses/paa/sympy-notebook/an-introduction-to-sympy.ipynb?flush\_cache=true}} 
    through \emph{Jupyter} viewer.

\end{description}

\section*{Altra attivit\`a scientifica}

\begin{description}

\item[January 2016: working on \emph{The Reasoned Schemer}]
    Read and work through the book \emph{The Reasoned Schemer}
    \footnote{\url{https://mitpress.mit.edu/books/reasoned-schemer}}, trying examples
    and done additional tests to better understand the content.
    In order to understand the underlying system, namely \emph{miniKanren}
    \footnote{\url{http://minikanren.org/}}, a sound knowledge of Scheme macros
    is necessary, so I turn to them as well. Actually, this study is
    going on up to now.

\item[January 2016: recurrences unfolding in SymPy] 
    Implementation of a Python module which allows us to unfold
    recurrence relations, built on top of SymPy. The following are basics idea underlying
    this module:
    \begin{itemize}
        \item understand recurrence's left hand side as a rewriting rule;
        \item do pattern matching on the \emph{whole} recurrence's left hand side,
            in order to proper unfold subterms belonging to recurrence's right hand
            side. This is an enhancement respect plain substitution, which eventually
            happen if whole pattern matching fails; 
        \item handle recurrences with both forwards and backwards subscripts on the
            left hand side;
        \item repeat unfolding up to a desired number of times, both using first order
            (namely, unfolding is always perfomed using the main recursion schema)
            and second order unfolding (namely, each intermediate unfold result is used
            in the next unfolding);
        \item do base instantiation in order to create a \emph{value-instantiated}
            recurrence. This allows us to \emph{evaluate} the initial segment of
            recurrence coefficients sequence, and in the case of recurrences
            depending on more than one previous value, to build triangle-shaped characterization;
        \item build \LaTeX\, code to produce pretty printed output.
    \end{itemize}
    An IPython notebook with some application of the implementation to known recurrences is available on-line 
    \footnote{\url{http://nbviewer.jupyter.org/github/massimo-nocentini/PhD/blob/master/courses/paa/sympy-notebook/recurrences-unfolding.ipynb?flush_cache=true}}.

\end{description}


\section*{Eventuali pubblicazioni o articoli inviati}

\begin{description}


\item[Pubblicazioni] 


\item[Articoli inviati]





\end{description}




\section*{Gruppo di ricerca e fondi ai quali afferisce}

Affiliato al Dipartimento di Statistica, Informatica, Applicazioni
``G. Parenti'' (DiSIA).

\begin{description}

\item[Gruppo INdAM] %inserire GNAMPA, GNCS, GNFM o GNSAGA

\item[Ex 60\%] Responsabile %inserire nome responsabile

\item[PRIN] Titolo\hskip 2cm Responsabile %inserire nome responsabile; 


\end{description}


\vfill

\end{document}


