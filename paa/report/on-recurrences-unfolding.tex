% article example for classicthesis.sty
\documentclass[a4paper,dottedtoc,headinclude,footinclude]{report} % KOMA-Script article scrartcl

\usepackage[pdfspacing,beramono,eulermath]{classicthesis} % nochapters
\usepackage{arsclassica}
\usepackage{lipsum}
\usepackage{url}
\usepackage{graphicx}
\usepackage{float}
\usepackage{amsthm, latexsym}
\usepackage{amssymb}
\usepackage{bm}
\usepackage{caption}
\usepackage{marginnote}
\usepackage{bbold}
\usepackage{minted}
\usepackage[T1]{fontenc} % T2A for cyrillics
\usepackage[utf8]{inputenc}				

\newcommand{\R}{{\mathbb R}}
\newcommand{\Q}{{\mathbb Q}}
\newcommand{\C}{{\mathbb C}}
\newcommand{\N}{{\mathbb N}}
\newcommand{\Z}{{\mathbb Z}}

\theoremstyle{plain}
%\numberwithin{equation}{section}
\newtheorem{thm}{Theorem}[section]
\newtheorem{theorem}[thm]{Theorem}
\newtheorem{lemma}[thm]{Lemma}
\newtheorem{example}[thm]{Example}
\newtheorem{definition}[thm]{Definition}
\newtheorem{proposition}[thm]{Proposition}

\begin{document}

    \title{\rmfamily\normalfont\spacedallcaps{On recurrences unfolding}}
    \author{\spacedlowsmallcaps{Massimo Nocentini}}
    \date{\today} 
    
    \maketitle
    
    \begin{abstract}
        \noindent In this document we show an implementation of a prototype 
        using the \emph{Python} language, on top of \emph{SymPy} module.
        It allows us to \emph{unfold} recurrence relations, where there
        is \emph{one} free variable. We show some applications to recurrences
        describing average numbers of Quicksort's checks and swaps; moreover, 
        possibly new characterizations for Fibonacci numbers are given.
    \end{abstract}
       
    \tableofcontents
    
    \chapter{Quicksort recurrences}

    \section{On the average number of checks}
    Let $c_{n}$ be the number of checks occuring on the computation
    of Quicksort algorithm on a vector of length $n$. After some manipulation
    starting from a general relation over $c_{n}$, the following
    recurrence relation can be derived:
    \begin{displaymath}
        \frac{c_{n}}{n + 1} = \frac{2}{n + 1} + \frac{1}{n} c_{n - 1}
    \end{displaymath}
    here is a terms cache:
    \begin{center}
    \begin{displaymath}
        \left \{ \frac{1}{n} c_{n - 1} : \frac{c_{n - 2}}{n - 1} + \frac{2}{n}, \quad \frac{c_{n - 4}}{n - 3} : \frac{2}{n - 3} + \frac{c_{n - 5}}{n - 4}, \quad \frac{c_{n - 3}}{n - 2} : \frac{2}{n - 2} + \frac{c_{n - 4}}{n - 3}, \quad \frac{c_{n - 2}}{n - 1} : \frac{2}{n - 1} + \frac{c_{n - 3}}{n - 2}\right \}
    \end{displaymath}
    \end{center}

    \subsection{A Subsection}
    \lipsum[1]

    \section{A Section}
    \lipsum[1]
    
    \subsection{A Subsection}
    \lipsum[1]
    \subsection{A Subsection}
    \lipsum[1]

    \iffalse
    % bib stuff
    \nocite{*}
    \addtocontents{toc}{\protect\vspace{\beforebibskip}}
    \addcontentsline{toc}{section}{\refname}    
    \bibliographystyle{plain}
    \bibliography{../Bibliography}
    \fi

\end{document}
