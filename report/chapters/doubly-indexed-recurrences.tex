
\chapter{Doubly indexed recurrences}

%\section{Splitting $\mathcal{C}$, the Catalan triangle}

A Riordan array $\mathcal{R}$ is usually denoted by a pair
of functions $(d(t), h(t))$, with the main request $h(0)=0$; 
moreover, an array $\mathcal{R}$ is \emph{proper} if $d(0)\neq 0$
and $\left.\frac{\partial h(t)}{\partial t}\right|_{t=0}\neq0$.
It is possible to characterize coefficients lying on column $k$ 
-- indexing is $0$-based, conventionally -- with the following
gf:
\begin{displaymath}
    d(t)h(t)^{k}
\end{displaymath}

Assume to start with a purely generic matrix:
\begin{equation}
\left[\begin{matrix}
c_{0,0} &  &  &  &  &  &  &  &  &  \\
c_{1,0} & c_{1,1} &  &  &  &  &  &  &  &  \\
c_{2,0} & c_{2,1} & c_{2,2} &  &  &  &  &  &  &  \\
c_{3,0} & c_{3,1} & c_{3,2} & c_{3,3} &  &  &  &  &  &  \\
c_{4,0} & c_{4,1} & c_{4,2} & c_{4,3} & c_{4,4} &  &  &  &  &  \\
c_{5,0} & c_{5,1} & c_{5,2} & c_{5,3} & c_{5,4} & c_{5,5} &  &  &  &  \\
c_{6,0} & c_{6,1} & c_{6,2} & c_{6,3} & c_{6,4} & c_{6,5} & c_{6,6} &  &  &  \\
c_{7,0} & c_{7,1} & c_{7,2} & c_{7,3} & c_{7,4} & c_{7,5} & c_{7,6} & c_{7,7} &  &  \\
c_{8,0} & c_{8,1} & c_{8,2} & c_{8,3} & c_{8,4} & c_{8,5} & c_{8,6} & c_{8,7} & c_{8,8} &  \\
c_{9,0} & c_{9,1} & c_{9,2} & c_{9,3} & c_{9,4} & c_{9,5} & c_{9,6} & c_{9,7} & c_{9,8} & c_{9,9} \\
\end{matrix}\right]
\label{eq:purely:generic:catalan}
\end{equation}
and unfold each term starting from row $1$ according to the Catalan recurrence:
\begin{equation}
c_{n + 1,k + 1} = c_{n,k} + c_{n,k + 1} + c_{n,k + 2} + c_{n,k + 3} + c_{n,k + 4} + c_{n,k + 5} + c_{n,k + 6} + c_{n,k + 7} + c_{n,k + 8} + c_{n,k + 9}
\label{eq:catalan:rec}
\end{equation}
the result is the following matrix, depending only on $c_{0,0}$, \textcolor{blue}{blue} colored:
\begin{equation}
\left[\begin{matrix}
\textcolor{blue}{c_{0,0}} &  &  &  &  &  &  &  &  &  \\
c_{0,0} & c_{0,0} &  &  &  &  &  &  &  &  \\
2 c_{0,0} & 2 c_{0,0} & c_{0,0} &  &  &  &  &  &  &  \\
5 c_{0,0} & 5 c_{0,0} & 3 c_{0,0} & c_{0,0} &  &  &  &  &  &  \\
14 c_{0,0} & 14 c_{0,0} & 9 c_{0,0} & 4 c_{0,0} & c_{0,0} &  &  &  &  &  \\
42 c_{0,0} & 42 c_{0,0} & 28 c_{0,0} & 14 c_{0,0} & 5 c_{0,0} & c_{0,0} &  &  &  &  \\
132 c_{0,0} & 132 c_{0,0} & 90 c_{0,0} & 48 c_{0,0} & 20 c_{0,0} & 6 c_{0,0} & c_{0,0} &  &  &  \\
429 c_{0,0} & 429 c_{0,0} & 297 c_{0,0} & 165 c_{0,0} & 75 c_{0,0} & 27 c_{0,0} & 7 c_{0,0} & c_{0,0} &  &  \\
1430 c_{0,0} & 1430 c_{0,0} & 1001 c_{0,0} & 572 c_{0,0} & 275 c_{0,0} & 110 c_{0,0} & 35 c_{0,0} & 8 c_{0,0} & c_{0,0} &  \\
4862 c_{0,0} & 4862 c_{0,0} & 3432 c_{0,0} & 2002 c_{0,0} & 1001 c_{0,0} & 429 c_{0,0} & 154 c_{0,0} & 44 c_{0,0} & 9 c_{0,0} & c_{0,0} \\
\end{matrix}\right]
\label{eq:unfolded:catalan}
\end{equation}
to obtain a standard triangle of numbers, known as Catalan triangle, just plug $c_{0,0}=1$
in the above matrix:
\begin{equation}
\left[\begin{matrix}
1 &  &  &  &  &  &  &  &  &  \\
1 & 1 &  &  &  &  &  &  &  &  \\
2 & 2 & 1 &  &  &  &  &  &  &  \\
5 & 5 & 3 & 1 &  &  &  &  &  &  \\
14 & 14 & 9 & 4 & 1 &  &  &  &  &  \\
42 & 42 & 28 & 14 & 5 & 1 &  &  &  &  \\
132 & 132 & 90 & 48 & 20 & 6 & 1 &  &  &  \\
429 & 429 & 297 & 165 & 75 & 27 & 7 & 1 &  &  \\
1430 & 1430 & 1001 & 572 & 275 & 110 & 35 & 8 & 1 &  \\
4862 & 4862 & 3432 & 2002 & 1001 & 429 & 154 & 44 & 9 & 1 \\
\end{matrix}\right]
\label{eq:standard:catalan}
\end{equation}
Such triangle is the matrix expansion of the Riordan array $\mathcal{C}$ defined as:
\begin{displaymath}
    \mathcal{C}=\left(\frac{1-\sqrt{1-4\,t}}{2\,t},
        \frac{1-\sqrt{1-4\,t}}{2}\right)
\end{displaymath}

\subsection{Two-way splitting}

Now let us unfold starting from row $2$, leaving the first \emph{two} rows purely generic, 
producing matrix in \autoref{eq:two:splitted:catalan}. We see that the lower part of the
matrix -- coefficients lying under row $2$, such row included -- dependends on $\textcolor{blue}{c_{1,0}}$
and $\textcolor{red}{c_{1,1}}$: but it should be possible to rewrite $\textcolor{red}{c_{1,1}}$
according \autoref{eq:catalan:rec}, yielding matrix in \autoref{eq:clean:two:splitted:catalan},
which depends only on $\textcolor{blue}{c_{0,0}}$ and $\textcolor{blue}{c_{1,0}}$.
\begin{sidewaystable}
%\vskip-4cm
\footnotesize
%\rotatebox{90}{$
\begin{align}
\left[\begin{matrix}
\textcolor{blue}{c_{0,0}} &  &  &  &  &  &  &  &  &  \\
\textcolor{blue}{c_{1,0}} & \textcolor{red}{c_{1,1}} &  &  &  &  &  &  &  &  \\
c_{1,0} + c_{1,1} & c_{1,0} + c_{1,1} & c_{1,1} &  &  &  &  &  &  &  \\
2 c_{1,0} + 3 c_{1,1} & 2 c_{1,0} + 3 c_{1,1} & c_{1,0} + 2 c_{1,1} & c_{1,1} &  &  &  &  &  &  \\
5 c_{1,0} + 9 c_{1,1} & 5 c_{1,0} + 9 c_{1,1} & 3 c_{1,0} + 6 c_{1,1} & c_{1,0} + 3 c_{1,1} & c_{1,1} &  &  &  &  &  \\
14 c_{1,0} + 28 c_{1,1} & 14 c_{1,0} + 28 c_{1,1} & 9 c_{1,0} + 19 c_{1,1} & 4 c_{1,0} + 10 c_{1,1} & c_{1,0} + 4 c_{1,1} & c_{1,1} &  &  &  &  \\
42 c_{1,0} + 90 c_{1,1} & 42 c_{1,0} + 90 c_{1,1} & 28 c_{1,0} + 62 c_{1,1} & 14 c_{1,0} + 34 c_{1,1} & 5 c_{1,0} + 15 c_{1,1} & c_{1,0} + 5 c_{1,1} & c_{1,1} &  &  &  \\
132 c_{1,0} + 297 c_{1,1} & 132 c_{1,0} + 297 c_{1,1} & 90 c_{1,0} + 207 c_{1,1} & 48 c_{1,0} + 117 c_{1,1} & 20 c_{1,0} + 55 c_{1,1} & 6 c_{1,0} + 21 c_{1,1} & c_{1,0} + 6 c_{1,1} & c_{1,1} &  &  \\
429 c_{1,0} + 1001 c_{1,1} & 429 c_{1,0} + 1001 c_{1,1} & 297 c_{1,0} + 704 c_{1,1} & 165 c_{1,0} + 407 c_{1,1} & 75 c_{1,0} + 200 c_{1,1} & 27 c_{1,0} + 83 c_{1,1} & 7 c_{1,0} + 28 c_{1,1} & c_{1,0} + 7 c_{1,1} & c_{1,1} &  \\
1430 c_{1,0} + 3432 c_{1,1} & 1430 c_{1,0} + 3432 c_{1,1} & 1001 c_{1,0} + 2431 c_{1,1} & 572 c_{1,0} + 1430 c_{1,1} & 275 c_{1,0} + 726 c_{1,1} & 110 c_{1,0} + 319 c_{1,1} & 35 c_{1,0} + 119 c_{1,1} & 8 c_{1,0} + 36 c_{1,1} & c_{1,0} + 8 c_{1,1} & c_{1,1} \\
\end{matrix}\right]
\label{eq:two:splitted:catalan} \\
\left[\begin{matrix}
\textcolor{blue}{c_{0,0}} &  &  &  &  &  &  &  &  &  \\
\textcolor{blue}{c_{1,0}} & c_{0,0} &  &  &  &  &  &  &  &  \\
c_{0,0} + c_{1,0} & c_{0,0} + c_{1,0} & c_{0,0} &  &  &  &  &  &  &  \\
3 c_{0,0} + 2 c_{1,0} & 3 c_{0,0} + 2 c_{1,0} & 2 c_{0,0} + c_{1,0} & c_{0,0} &  &  &  &  &  &  \\
9 c_{0,0} + 5 c_{1,0} & 9 c_{0,0} + 5 c_{1,0} & 6 c_{0,0} + 3 c_{1,0} & 3 c_{0,0} + c_{1,0} & c_{0,0} &  &  &  &  &  \\
28 c_{0,0} + 14 c_{1,0} & 28 c_{0,0} + 14 c_{1,0} & 19 c_{0,0} + 9 c_{1,0} & 10 c_{0,0} + 4 c_{1,0} & 4 c_{0,0} + c_{1,0} & c_{0,0} &  &  &  &  \\
90 c_{0,0} + 42 c_{1,0} & 90 c_{0,0} + 42 c_{1,0} & 62 c_{0,0} + 28 c_{1,0} & 34 c_{0,0} + 14 c_{1,0} & 15 c_{0,0} + 5 c_{1,0} & 5 c_{0,0} + c_{1,0} & c_{0,0} &  &  &  \\
297 c_{0,0} + 132 c_{1,0} & 297 c_{0,0} + 132 c_{1,0} & 207 c_{0,0} + 90 c_{1,0} & 117 c_{0,0} + 48 c_{1,0} & 55 c_{0,0} + 20 c_{1,0} & 21 c_{0,0} + 6 c_{1,0} & 6 c_{0,0} + c_{1,0} & c_{0,0} &  &  \\
1001 c_{0,0} + 429 c_{1,0} & 1001 c_{0,0} + 429 c_{1,0} & 704 c_{0,0} + 297 c_{1,0} & 407 c_{0,0} + 165 c_{1,0} & 200 c_{0,0} + 75 c_{1,0} & 83 c_{0,0} + 27 c_{1,0} & 28 c_{0,0} + 7 c_{1,0} & 7 c_{0,0} + c_{1,0} & c_{0,0} &  \\
3432 c_{0,0} + 1430 c_{1,0} & 3432 c_{0,0} + 1430 c_{1,0} & 2431 c_{0,0} + 1001 c_{1,0} & 1430 c_{0,0} + 572 c_{1,0} & 726 c_{0,0} + 275 c_{1,0} & 319 c_{0,0} + 110 c_{1,0} & 119 c_{0,0} + 35 c_{1,0} & 36 c_{0,0} + 8 c_{1,0} & 8 c_{0,0} + c_{1,0} & c_{0,0} \\
\end{matrix}\right]
\label{eq:clean:two:splitted:catalan}
%$}
\end{align}
\end{sidewaystable}
Two-way unfolding suggests that we can characterize coefficients lying on a column $k$ as a 
combination of \emph{two} generating functions $a_{k}(t)$ and $b_{k}(t)$, multiplied by $c_{0,0}$
and $c_{1,0}$ respectively. Because coefficients lying on a column $k$ of a Riordan array
can also be characterized by the convolution $d(t)h(t)^{k}$, we can state the following relation:
\begin{equation} 
    d(t)h(t)^{k} = c_{0,0}a_{k}(t) + c_{1,0}b_{k}(t)
    \label{eq:two:splitted:catalan:generic:column:relation}
\end{equation} 
which can be used toward to assert a relation about two adjacent columns $k$ and $k+1$. 
To see this, starts by taking the ratio of those columns:
\begin{displaymath} 
    \frac{d(t)h(t)^{k+1}}{d(t)h(t)^{k}} = \frac{c_{0,0}a_{k+1}(t) + c_{1,0}b_{k+1}(t)}
        {c_{0,0}a_{k}(t) + c_{1,0}b_{k}(t)}
\end{displaymath} 
which can be simplified to:
\begin{displaymath} 
    h(t)\left(c_{0,0}a_{k}(t) + c_{1,0}b_{k}(t)\right) = 
        c_{0,0}a_{k+1}(t) + c_{1,0}b_{k+1}(t)
\end{displaymath} 
which holds if and only if:
\begin{displaymath} 
    h(t) a_{k}(t) = a_{k+1}(t) \quad \wedge \quad h(t) b_{k}(t) = b_{k+1}(t)
\end{displaymath} 
unfolding such relations we get, for a generic column $j$ and an integer $s\in\lbrace 0,\ldots,j\rbrace$:
\begin{displaymath} 
    a_{j}(t) = h(t)^{j-s} a_{s}(t) \quad \quad 
    b_{j}(t) = h(t)^{j-s} b_{s}(t)
\end{displaymath} 
fixing $s=0$ we go down to the very first column:
\begin{displaymath} 
    a_{j}(t) = h(t)^{j} a_{0}(t) \quad \quad 
    b_{j}(t) = h(t)^{j} b_{0}(t)
\end{displaymath} 
therefore there exists two Riordan arrays $\mathcal{C}_{a}$ and $\mathcal{C}_{b}$
defined as follows:
\begin{displaymath} 
    \mathcal{C}_{a} = \left(a_{0}(t), h(t)\right) \quad \quad 
    \mathcal{C}_{b} = \left(b_{0}(t), h(t)\right) \quad \quad 
\end{displaymath} 
such that:
\begin{equation} 
    \mathcal{C} = c_{0,0}\mathcal{C}_{a} + c_{1,0}\mathcal{C}_{b}
    \label{eq:two:splitted:catalan:riordan:expansion}
\end{equation} 
From now on, we attach a super-script $^{(i)}$, for $i\in\lbrace 2,3\ldots\rbrace$, 
to functions used in a sum expansion in order to be clear that such functions 
refers to a matrix unfolding where $i$ variables, namely $c_{0,0},\ldots,c_{i-1,0}$, 
are present. In the case above, $a_{0}^{(2)}(t)$ and $b_{0}^{(2)}(t)$ 
denote the same functions as $a_{0}(t)$ and $b_{0}(t)$.


Studying \autoref{eq:two:splitted:catalan:generic:column:relation}
with $k=0$ we get:
\begin{equation} 
    d(t) = c_{0,0}a_{0}^{(2)}(t) + c_{1,0}b_{0}^{(2)}(t)
\end{equation} 
a look at \autoref{eq:standard:catalan} allows us to set both $c_{0,0}$ and $c_{1,0}$ to $1$,
and looking at \autoref{eq:clean:two:splitted:catalan} it seems to be permitted to set 
function $b_{0}^{(2)}(t)$ to $t\,d(t)$; therefore, 
\begin{equation} 
    d(t) = a_{0}^{(2)}(t) + t\,d(t) \rightarrow d(t)(1-t) = a_{0}^{(2)}(t)
\end{equation} 
holds. So \autoref{eq:two:splitted:catalan:riordan:expansion} can be rewritten as follows:
\begin{displaymath} 
    \mathcal{C}(d(t), h(t)) = \mathcal{C}_{a}(d(t)(1-t), h(t)) + \mathcal{C}_{b}(t\,d(t), h(t))
\end{displaymath} 
in expanded matrix notation:
\begin{equation}
\hspace{-3cm}
\mathcal{C} = 
\left[\begin{matrix}
1 &  &  &  &  &  &  &  &  &  \\
0 & 1 &  &  &  &  &  &  &  &  \\
1 & 1 & 1 &  &  &  &  &  &  &  \\
3 & 3 & 2 & 1 &  &  &  &  &  &  \\
9 & 9 & 6 & 3 & 1 &  &  &  &  &  \\
28 & 28 & 19 & 10 & 4 & 1 &  &  &  &  \\
90 & 90 & 62 & 34 & 15 & 5 & 1 &  &  &  \\
297 & 297 & 207 & 117 & 55 & 21 & 6 & 1 &  &  \\
1001 & 1001 & 704 & 407 & 200 & 83 & 28 & 7 & 1 &  \\
3432 & 3432 & 2431 & 1430 & 726 & 319 & 119 & 36 & 8 & 1 \\
\end{matrix}\right] +
\left[\begin{matrix}
0 &  &  &  &  &  &  &  &  &  \\
1 & 0 &  &  &  &  &  &  &  &  \\
1 & 1 & 0 &  &  &  &  &  &  &  \\
2 & 2 & 1 & 0 &  &  &  &  &  &  \\
5 & 5 & 3 & 1 & 0 &  &  &  &  &  \\
14 & 14 & 9 & 4 & 1 & 0 &  &  &  &  \\
42 & 42 & 28 & 14 & 5 & 1 & 0 &  &  &  \\
132 & 132 & 90 & 48 & 20 & 6 & 1 & 0 &  &  \\
429 & 429 & 297 & 165 & 75 & 27 & 7 & 1 & 0 &  \\
1430 & 1430 & 1001 & 572 & 275 & 110 & 35 & 8 & 1 & 0 \\
\end{matrix}\right]\\ 
\label{eq:matrix:expansion:two:splitted:catalan}
\end{equation}


\subsection{Three-way splitting}

Now we're going to unfold matrix in \autoref{eq:purely:generic:catalan} starting from row $3$,
aiming to rewrite that matrix such that it depends only on $c_{0,0}, c_{1,0}, c_{2,0}$.
We can characterize the convolution for the generic column $k$ using three
functions $a_{k}^{(3)}(t), b_{k}^{(3)}(t), c_{k}^{(3)}(t)$ as follows:
\begin{equation} 
    d(t)h(t)^{k} = c_{0,0}a_{k}^{(3)}(t) + c_{1,0}b_{k}^{(3)}(t) + c_{2,0}c_{k}^{(3)}(t)
    \label{eq:three:splitted:catalan:generic:column:relation}
\end{equation} 
In the style of the argument shown in the previous section, we seek for:
\begin{equation} 
    \mathcal{C} = c_{0,0}\mathcal{C}_{a} + c_{1,0}\mathcal{C}_{b} + c_{2,0}\mathcal{C}_{c}
    \label{eq:three:splitted:catalan:riordan:expansion}
\end{equation} 
where:
\begin{displaymath} 
    \mathcal{C}_{a} = \left(a_{0}^{(3)}(t), h(t)\right) \quad \quad 
    \mathcal{C}_{b} = \left(b_{0}^{(3)}(t), h(t)\right) \quad \quad 
    \mathcal{C}_{c} = \left(c_{0}^{(3)}(t), h(t)\right) \quad \quad 
\end{displaymath} 
Looking at \autoref{eq:clean:three:splitted:catalan} it seems that the following relations hold:
\begin{displaymath} 
    b_{0}^{(3)}(t) = t\,a_{0}^{(2)}(t) = t(1-t)d(t)\quad \quad 
    c_{0}^{(3)}(t) = t\,b_{0}^{(2)}(t) = t^{2}d(t) 
\end{displaymath} 
and \autoref{eq:standard:catalan} allows us to set $c_{2,0}$ to $2$. Sustituting in
\autoref{eq:three:splitted:catalan:generic:column:relation}, with $k=0$, yields
function $a_{0}^{(3)}(t)$:
\begin{displaymath} 
    d(t)\left(1 -t(1-t) -2t^{2}\right) = a_{0}^{(3)}(t)
\end{displaymath} 
so:
\begin{displaymath} 
    d(t)\left(1 -t -t^{2}\right) = a_{0}^{(3)}(t)
\end{displaymath} 
therefore $\mathcal{C}$ factorization, using $3$ variables, can be written as:
\begin{displaymath} 
    \mathcal{C} = 
        \left(\left(1 -t -t^{2}\right)d(t), h(t)\right) +
        \left(\left(t-t^{2}\right)d(t), h(t)\right) +
        2\,\left(t^{2}d(t), h(t)\right) 
\end{displaymath} 
in matrix notation:
\begin{displaymath}
\hspace{-3cm}
\scriptsize
\mathcal{C} = 
\left[\begin{matrix}
1 &  &  &  &  &  &  &  &  &  \\
0 & 1 &  &  &  &  &  &  &  &  \\
0 & 1 & 1 &  &  &  &  &  &  &  \\
2 & 2 & 2 & 1 &  &  &  &  &  &  \\
7 & 7 & 5 & 3 & 1 &  &  &  &  &  \\
23 & 23 & 16 & 9 & 4 & 1 &  &  &  &  \\
76 & 76 & 53 & 30 & 14 & 5 & 1 &  &  &  \\
255 & 255 & 179 & 103 & 50 & 20 & 6 & 1 &  &  \\
869 & 869 & 614 & 359 & 180 & 77 & 27 & 7 & 1 &  \\
3003 & 3003 & 2134 & 1265 & 651 & 292 & 112 & 35 & 8 & 1 \\
\end{matrix}\right] + 
\left[\begin{matrix}
0 &  &  &  &  &  &  &  &  &  \\
1 & 0 &  &  &  &  &  &  &  &  \\
0 & 1 & 0 &  &  &  &  &  &  &  \\
1 & 1 & 1 & 0 &  &  &  &  &  &  \\
3 & 3 & 2 & 1 & 0 &  &  &  &  &  \\
9 & 9 & 6 & 3 & 1 & 0 &  &  &  &  \\
28 & 28 & 19 & 10 & 4 & 1 & 0 &  &  &  \\
90 & 90 & 62 & 34 & 15 & 5 & 1 & 0 &  &  \\
297 & 297 & 207 & 117 & 55 & 21 & 6 & 1 & 0 &  \\
1001 & 1001 & 704 & 407 & 200 & 83 & 28 & 7 & 1 & 0 \\
\end{matrix}\right] 
\end{displaymath}
\begin{displaymath}
\scriptsize
+ 2\,\left[\begin{matrix}
0 &  &  &  &  &  &  &  &  &  \\
0 & 0 &  &  &  &  &  &  &  &  \\
1 & 0 & 0 &  &  &  &  &  &  &  \\
1 & 1 & 0 & 0 &  &  &  &  &  &  \\
2 & 2 & 1 & 0 & 0 &  &  &  &  &  \\
5 & 5 & 3 & 1 & 0 & 0 &  &  &  &  \\
14 & 14 & 9 & 4 & 1 & 0 & 0 &  &  &  \\
42 & 42 & 28 & 14 & 5 & 1 & 0 & 0 &  &  \\
132 & 132 & 90 & 48 & 20 & 6 & 1 & 0 & 0 &  \\
429 & 429 & 297 & 165 & 75 & 27 & 7 & 1 & 0 & 0 \\ 
\end{matrix}\right]
\end{displaymath}
\begin{sidewaystable}
%\vskip-4cm
\scriptsize
\begin{equation}
\left[\begin{matrix}
\textcolor{blue}{c_{0,0}} &  &  &  &  &  &  &  &  &  \\
\textcolor{blue}{c_{1,0}} & \textcolor{red}{c_{1,1}} &  &  &  &  &  &  &  &  \\
\textcolor{blue}{c_{2,0}} & \textcolor{red}{c_{2,1}} & \textcolor{red}{c_{2,2}} &  &  &  &  &  &  &  \\
c_{2,0} + c_{2,1} + c_{2,2} & c_{2,0} + c_{2,1} + c_{2,2} & c_{2,1} + c_{2,2} & c_{2,2} &  &  &  &  &  &  \\
2 c_{2,0} + 3 c_{2,1} + 4 c_{2,2} & 2 c_{2,0} + 3 c_{2,1} + 4 c_{2,2} & c_{2,0} + 2 c_{2,1} + 3 c_{2,2} & c_{2,1} + 2 c_{2,2} & c_{2,2} &  &  &  &  &  \\
5 c_{2,0} + 9 c_{2,1} + 14 c_{2,2} & 5 c_{2,0} + 9 c_{2,1} + 14 c_{2,2} & 3 c_{2,0} + 6 c_{2,1} + 10 c_{2,2} & c_{2,0} + 3 c_{2,1} + 6 c_{2,2} & c_{2,1} + 3 c_{2,2} & c_{2,2} &  &  &  &  \\
14 c_{2,0} + 28 c_{2,1} + 48 c_{2,2} & 14 c_{2,0} + 28 c_{2,1} + 48 c_{2,2} & 9 c_{2,0} + 19 c_{2,1} + 34 c_{2,2} & 4 c_{2,0} + 10 c_{2,1} + 20 c_{2,2} & c_{2,0} + 4 c_{2,1} + 10 c_{2,2} & c_{2,1} + 4 c_{2,2} & c_{2,2} &  &  &  \\
42 c_{2,0} + 90 c_{2,1} + 165 c_{2,2} & 42 c_{2,0} + 90 c_{2,1} + 165 c_{2,2} & 28 c_{2,0} + 62 c_{2,1} + 117 c_{2,2} & 14 c_{2,0} + 34 c_{2,1} + 69 c_{2,2} & 5 c_{2,0} + 15 c_{2,1} + 35 c_{2,2} & c_{2,0} + 5 c_{2,1} + 15 c_{2,2} & c_{2,1} + 5 c_{2,2} & c_{2,2} &  &  \\
132 c_{2,0} + 297 c_{2,1} + 572 c_{2,2} & 132 c_{2,0} + 297 c_{2,1} + 572 c_{2,2} & 90 c_{2,0} + 207 c_{2,1} + 407 c_{2,2} & 48 c_{2,0} + 117 c_{2,1} + 242 c_{2,2} & 20 c_{2,0} + 55 c_{2,1} + 125 c_{2,2} & 6 c_{2,0} + 21 c_{2,1} + 56 c_{2,2} & c_{2,0} + 6 c_{2,1} + 21 c_{2,2} & c_{2,1} + 6 c_{2,2} & c_{2,2} &  \\
429 c_{2,0} + 1001 c_{2,1} + 2002 c_{2,2} & 429 c_{2,0} + 1001 c_{2,1} + 2002 c_{2,2} & 297 c_{2,0} + 704 c_{2,1} + 1430 c_{2,2} & 165 c_{2,0} + 407 c_{2,1} + 858 c_{2,2} & 75 c_{2,0} + 200 c_{2,1} + 451 c_{2,2} & 27 c_{2,0} + 83 c_{2,1} + 209 c_{2,2} & 7 c_{2,0} + 28 c_{2,1} + 84 c_{2,2} & c_{2,0} + 7 c_{2,1} + 28 c_{2,2} & c_{2,1} + 7 c_{2,2} & c_{2,2} \\
\end{matrix}\right]
\label{eq:three:splitted:catalan}
\end{equation}
\begin{equation}
\left[\begin{matrix}
\textcolor{blue}{c_{0,0}} &  &  &  &  &  &  &  &  &  \\
\textcolor{blue}{c_{1,0}} & c_{0,0} &  &  &  &  &  &  &  &  \\
\textcolor{blue}{c_{2,0}} & c_{0,0} + c_{1,0} & c_{0,0} &  &  &  &  &  &  &  \\
2 c_{0,0} + c_{1,0} + c_{2,0} & 2 c_{0,0} + c_{1,0} + c_{2,0} & 2 c_{0,0} + c_{1,0} & c_{0,0} &  &  &  &  &  &  \\
7 c_{0,0} + 3 c_{1,0} + 2 c_{2,0} & 7 c_{0,0} + 3 c_{1,0} + 2 c_{2,0} & 5 c_{0,0} + 2 c_{1,0} + c_{2,0} & 3 c_{0,0} + c_{1,0} & c_{0,0} &  &  &  &  &  \\
23 c_{0,0} + 9 c_{1,0} + 5 c_{2,0} & 23 c_{0,0} + 9 c_{1,0} + 5 c_{2,0} & 16 c_{0,0} + 6 c_{1,0} + 3 c_{2,0} & 9 c_{0,0} + 3 c_{1,0} + c_{2,0} & 4 c_{0,0} + c_{1,0} & c_{0,0} &  &  &  &  \\
76 c_{0,0} + 28 c_{1,0} + 14 c_{2,0} & 76 c_{0,0} + 28 c_{1,0} + 14 c_{2,0} & 53 c_{0,0} + 19 c_{1,0} + 9 c_{2,0} & 30 c_{0,0} + 10 c_{1,0} + 4 c_{2,0} & 14 c_{0,0} + 4 c_{1,0} + c_{2,0} & 5 c_{0,0} + c_{1,0} & c_{0,0} &  &  &  \\
255 c_{0,0} + 90 c_{1,0} + 42 c_{2,0} & 255 c_{0,0} + 90 c_{1,0} + 42 c_{2,0} & 179 c_{0,0} + 62 c_{1,0} + 28 c_{2,0} & 103 c_{0,0} + 34 c_{1,0} + 14 c_{2,0} & 50 c_{0,0} + 15 c_{1,0} + 5 c_{2,0} & 20 c_{0,0} + 5 c_{1,0} + c_{2,0} & 6 c_{0,0} + c_{1,0} & c_{0,0} &  &  \\
869 c_{0,0} + 297 c_{1,0} + 132 c_{2,0} & 869 c_{0,0} + 297 c_{1,0} + 132 c_{2,0} & 614 c_{0,0} + 207 c_{1,0} + 90 c_{2,0} & 359 c_{0,0} + 117 c_{1,0} + 48 c_{2,0} & 180 c_{0,0} + 55 c_{1,0} + 20 c_{2,0} & 77 c_{0,0} + 21 c_{1,0} + 6 c_{2,0} & 27 c_{0,0} + 6 c_{1,0} + c_{2,0} & 7 c_{0,0} + c_{1,0} & c_{0,0} &  \\
3003 c_{0,0} + 1001 c_{1,0} + 429 c_{2,0} & 3003 c_{0,0} + 1001 c_{1,0} + 429 c_{2,0} & 2134 c_{0,0} + 704 c_{1,0} + 297 c_{2,0} & 1265 c_{0,0} + 407 c_{1,0} + 165 c_{2,0} & 651 c_{0,0} + 200 c_{1,0} + 75 c_{2,0} & 292 c_{0,0} + 83 c_{1,0} + 27 c_{2,0} & 112 c_{0,0} + 28 c_{1,0} + 7 c_{2,0} & 35 c_{0,0} + 7 c_{1,0} + c_{2,0} & 8 c_{0,0} + c_{1,0} & c_{0,0} \\
\end{matrix}\right]
\label{eq:clean:three:splitted:catalan}
\end{equation}
\end{sidewaystable}

\subsection{A possible generalization}

Arguments in previous sections, were unfolding leaving $2$ and $3$ variables are
described, allow us to conjecture a general factorization for a Riordan array $\mathcal{C}$.
Let $\lbrace \alpha_{k,i}^{(s)}(t)\rbrace_{k,i\in\mathbb{N}}$ be a family of generating functions
in the indeterminate $t$, related to matrix unfolding depending on $s$ free variables.
Then we can state the following relation about the convolution characterizing a 
generic column $k$ of $\mathcal{C}$ as follows:
\begin{equation}
    d(t)h(t)^{k} = c_{0,0}\alpha_{k,0}^{(s)}(t) + c_{1,0}\alpha_{k,1}^{(s)}(t) + \ldots
        c_{s-1,0}\alpha_{k,s-1}^{(s)}(t)
\end{equation}
setting $k=0$ we can get rid of function $h$, looking for a factorization of function $d$:
\begin{equation}
    d(t) = c_{0,0}\alpha_{0,0}^{(s)}(t) + c_{1,0}\alpha_{0,1}^{(s)}(t) + \ldots +
        c_{s-1,0}\alpha_{0,s-1}^{(s)}(t)
\end{equation}
where $\alpha_{0, j}^{(s)}(t) = t\,\alpha_{0, j-1}^{(s-1)}(t)$, for $j\in\lbrace 1,\ldots,s-1\rbrace$;
the base case for the recursion is $\alpha_{0, 1}^{(2)}(t) = t\,d(t)$.

In \autoref{eq:table:a:zero:zero:pascal}, \autoref{eq:table:a:zero:zero:shapiro} and
\autoref{eq:table:a:zero:zero:catalan}, we report three tables with schemata to compute
$\alpha_{0,0}^{(s)}(t)$, for $s\in\lbrace 2, 3, 4, 5\rbrace$ relative to 
Pascal triangle $\mathcal{P}$, Shapiro triangle $\mathcal{S}$ and, finally,
Catalan triangle $\mathcal{C}$, respectively. Recall the \emph{fundamental}
base case for the recursion $\alpha_{0,1}^{(2)}(t) = t\,d(t)$, then the following 
definitions for $d(t)$ functions and initial coefficients are used for each array: 
\begin{itemize}
\item $\mathcal{P}$:
\begin{displaymath}
     d(t) = \frac{1}{1-t} \quad \quad p_{i,0}=1, i\in\lbrace 0,\ldots,4\rbrace
\end{displaymath}

\item $\mathcal{S}$:
\begin{displaymath}
        d(t) = \frac{1-2t-\sqrt{1-4t}}{2t^{2}} \quad \quad 
            s_{0,0}=1\quad s_{1,0}=2\quad s_{2,0}=5\quad s_{3,0}=14\quad s_{4,0}=42
\end{displaymath}

\item $\mathcal{C}$:
\begin{displaymath}
        d(t) = \frac{1-\sqrt{1-4t}}{2t} \quad \quad 
            c_{0,0}=1\quad c_{1,0}=1\quad c_{2,0}=2\quad c_{3,0}=5\quad c_{4,0}=14
\end{displaymath}

\end{itemize}

\begin{sidewaystable}
\scriptsize
\begin{equation}
    %\hspace{-3cm}
    \begin{array}{ccc}
        s & d(t) & \alpha_{0,0}^{(s)}(t) \\
        \hline
        2 & p_{0,0}\alpha_{0,0}^{(2)}(t) + p_{1,0}\alpha_{0,1}^{(2)}(t) = \alpha_{0,0}^{(2)}(t) + t\,d(t) & 1 \\
        3 & p_{0,0}\alpha_{0,0}^{(3)}(t) + t\left(p_{1,0}\alpha_{0,0}^{(2)}(t) + p_{2,0}\alpha_{0,1}^{(2)}(t)\right) =
            \alpha_{0,0}^{(3)}(t) + t + t^{2}\,d(t) & 1 \\
        4 & p_{0,0}\alpha_{0,0}^{(4)}(t) + t\left(p_{1,0}\alpha_{0,0}^{(3)}(t) + p_{2,0}\alpha_{0,1}^{(3)}(t) + p_{3,0}\alpha_{0,2}^{(3)}(t)\right)  =
            \alpha_{0,0}^{(4)}(t) + t + t^{2} + t^{3}\,d(t) & 1 \\
        5 & p_{0,0}\alpha_{0,0}^{(5)}(t) + t\left(p_{1,0}\alpha_{0,0}^{(4)}(t) + p_{2,0}\alpha_{0,1}^{(4)}(t) + p_{3,0}\alpha_{0,2}^{(4)}(t) + p_{4,0}\alpha_{0,3}^{(4)}(t)\right)  =
            \alpha_{0,0}^{(5)}(t) + t + t^{2} + t^{3} + t^{4}\,d(t) & 1 \\
    \end{array}
    \label{eq:table:a:zero:zero:pascal}
\end{equation}

\begin{equation}
    %\hspace{-5cm}
    \begin{array}{ccc}
        s & d(t) & \alpha_{0,0}^{(s)}(t) \\
        \hline
        2 & s_{0,0}\alpha_{0,0}^{(2)}(t) + s_{1,0}\alpha_{0,1}^{(2)}(t) = \alpha_{0,0}^{(2)}(t) + 2t\,d(t) & d(t)(1-2t) \\
        3 & s_{0,0}\alpha_{0,0}^{(3)}(t) + t\left(s_{1,0}\alpha_{0,0}^{(2)}(t) + s_{2,0}\alpha_{0,1}^{(2)}(t)\right) =
            \alpha_{0,0}^{(3)}(t) + 2t(1-2t)d(t) + 5t^{2}\,d(t) & d(t)\left(1-2t-t^{2}\right) \\
        4 & s_{0,0}\alpha_{0,0}^{(4)}(t) + t\left(s_{1,0}\alpha_{0,0}^{(3)}(t) + s_{2,0}\alpha_{0,1}^{(3)}(t) + s_{3,0}\alpha_{0,2}^{(3)}(t)\right)  =
            \alpha_{0,0}^{(4)}(t) + 2t\,d(t)\left(1-2t-t^{2}\right) + 5t^{2}\,d(t)(1-2t) + 14t^{3}\,d(t) & d(t)\left(1-2t-t^{2}-2t^{3}\right) \\
        5 & s_{0,0}\alpha_{0,0}^{(5)}(t) + t\left(s_{1,0}\alpha_{0,0}^{(4)}(t) + s_{2,0}\alpha_{0,1}^{(4)}(t) + s_{3,0}\alpha_{0,2}^{(4)}(t) + s_{4,0}\alpha_{0,3}^{(4)}(t)\right)  =
            \alpha_{0,0}^{(5)}(t) + 2t\,d(t)\left(1-2t-t^{2}-2t^{3}\right) + 5t^{2}\,d(t)\left(1-2t-t^{2}\right) + 14t^{3}\,d(t)(1-2t) + 42t^{4}\,d(t) & d(t)\left(1-2t-t^{2}-2t^{3}-5t^{4}\right) \\
    \end{array}
    \label{eq:table:a:zero:zero:shapiro}
\end{equation}

% TODO: fix the following one, it is the same as the above one
\begin{equation}
    %\hspace{-5cm}
    \begin{array}{ccc}
        s & d(t) & \alpha_{0,0}^{(s)}(t) \\
        \hline
        2 & s_{0,0}\alpha_{0,0}^{(2)}(t) + s_{1,0}\alpha_{0,1}^{(2)}(t) = \alpha_{0,0}^{(2)}(t) + 2t\,d(t) & d(t)(1-2t) \\
        3 & s_{0,0}\alpha_{0,0}^{(3)}(t) + t\left(s_{1,0}\alpha_{0,0}^{(2)}(t) + s_{2,0}\alpha_{0,1}^{(2)}(t)\right) =
            \alpha_{0,0}^{(3)}(t) + 2t(1-2t)d(t) + 5t^{2}\,d(t) & d(t)\left(1-2t-t^{2}\right) \\
        4 & s_{0,0}\alpha_{0,0}^{(4)}(t) + t\left(s_{1,0}\alpha_{0,0}^{(3)}(t) + s_{2,0}\alpha_{0,1}^{(3)}(t) + s_{3,0}\alpha_{0,2}^{(3)}(t)\right)  =
            \alpha_{0,0}^{(4)}(t) + 2t\,d(t)\left(1-2t-t^{2}\right) + 5t^{2}\,d(t)(1-2t) + 14t^{3}\,d(t) & d(t)\left(1-2t-t^{2}-2t^{3}\right) \\
        5 & s_{0,0}\alpha_{0,0}^{(5)}(t) + t\left(s_{1,0}\alpha_{0,0}^{(4)}(t) + s_{2,0}\alpha_{0,1}^{(4)}(t) + s_{3,0}\alpha_{0,2}^{(4)}(t) + s_{4,0}\alpha_{0,3}^{(4)}(t)\right)  =
            \alpha_{0,0}^{(5)}(t) + 2t\,d(t)\left(1-2t-t^{2}-2t^{3}\right) + 5t^{2}\,d(t)\left(1-2t-t^{2}\right) + 14t^{3}\,d(t)(1-2t) + 42t^{4}\,d(t) & d(t)\left(1-2t-t^{2}-2t^{3}-5t^{4}\right) \\
    \end{array}
    \label{eq:table:a:zero:zero:catalan}
\end{equation}

\end{sidewaystable}










