
\chapter{On Fibonacci numbers}

\section{First Order unfolding}

We use the following recurrence to symbolically characterize Fibonacci numbers: 
\begin{equation}
    \label{eq:fib:first:order:unfolding}
    f_{n + 2} = f_{n} + f_{n + 1}
\end{equation}
In this section we are interested in applying \emph{first order} unfolding
to such relation, namely unfold both term $f_{n}$ and $f_{n+1}$ according
\autoref{eq:fib:first:order:unfolding}. Doing one unfolding step yield:
\begin{displaymath} 
    f_{n + 2} = f_{n - 2} + 2 f_{n - 1} + f_{n}
\end{displaymath}
Doing one more unfolding step yield:
\begin{displaymath} 
    f_{n + 2} = f_{n - 4} + 3 f_{n - 3} + 3 f_{n - 2} + f_{n - 1}
\end{displaymath}
let us continue to unfold up to $9$ times, collecting relations in 
\autoref{triangle:fib:first:order} and in \autoref{matrix:notation:triangle:fib:first:order}
using matrix notation. These tables allow us to derive the 
following generalization, which we state as a theorem.

\begin{thm} 
    Choose $m\in\mathbb{N}$. Let $f_{n}$ denote the $n$-th Fibonacci number,
    where $n \geq 2m$, then:
    \begin{displaymath} 
        f_{n} = \sum_{k=0}^{m} {{{m}\choose{k}} f_{n - 2 m + k}}
    \end{displaymath} 
\end{thm}
Preview theorem can be manipulated as follows:
\begin{displaymath} 
    \begin{split} 
        f_{n+m} &= \sum_{k\in\lbrace 0,\ldots,m\rbrace}{{{m}\choose{k}} f_{n - m + k}}\\
                &= \sum_{k\in\lbrace 0,\ldots,m\rbrace}{{{m}\choose{m-k}} f_{n - m + k}}\\
                &= \text{let } \lbrace j = m-k \rbrace\\
                &= \sum_{j\in\lbrace m,\ldots,0\rbrace}{{{m}\choose{j}} f_{n - j}}\\
    \end{split} 
\end{displaymath} 
as Benjamin and Quinn asks to show in an exercise.


\begin{sidewaystable}

    \begin{eqnarray}
        & f_{n + 2} = f_{n} + f_{n + 1} \\
        & f_{n + 2} = f_{n} + f_{n - 2} + 2 f_{n - 1}\\
        & f_{n + 2} = f_{n - 4} + 3 f_{n - 3} + 3 f_{n - 2} + f_{n - 1}\\
        & f_{n + 2} = f_{n - 6} + 4 f_{n - 5} + 6 f_{n - 4} + 4 f_{n - 3} + f_{n - 2}\\
        & f_{n + 2} = f_{n - 8} + 5 f_{n - 7} + 10 f_{n - 6} + 10 f_{n - 5} + 5 f_{n - 4} + f_{n - 3}\\
        & f_{n + 2} = f_{n - 10} + 6 f_{n - 9} + 15 f_{n - 8} + 20 f_{n - 7} + 15 f_{n - 6} + 6 f_{n - 5} + f_{n - 4}\\
        & f_{n + 2} = f_{n - 12} + 7 f_{n - 11} + 21 f_{n - 10} + 35 f_{n - 9} + 35 f_{n - 8} + 21 f_{n - 7} + 7 f_{n - 6} + f_{n - 5}\\
        & f_{n + 2} = f_{n - 14} + 8 f_{n - 13} + 28 f_{n - 12} + 56 f_{n - 11} + 70 f_{n - 10} + 56 f_{n - 9} + 28 f_{n - 8} + 8 f_{n - 7} + f_{n - 6}\\
        & f_{n + 2} = f_{n - 16} + 9 f_{n - 15} + 36 f_{n - 14} + 84 f_{n - 13} + 126 f_{n - 12} + 126 f_{n - 11} + 84 f_{n - 10} + 36 f_{n - 9} + 9 f_{n - 8} + f_{n - 7}\\
        & f_{n + 2} = f_{n - 18} + 10 f_{n - 17} + 45 f_{n - 16} + 120 f_{n - 15} + 210 f_{n - 14} + 252 f_{n - 13} + 210 f_{n - 12} + 120 f_{n - 11} + 45 f_{n - 10} + 10 f_{n - 9} + f_{n - 8}
        \end{eqnarray}

    \caption{Relations produced by $0$ up to $9$ \emph{first order} unfolding steps}
    \label{triangle:fib:first:order}
\end{sidewaystable}

\begin{sidewaystable}

    \begin{displaymath}
        \left[\begin{array}{ccccccccccccccccccccc}
            1 & 0 & 0 & 0 & 0 & 0 & 0 & 0 & 0 & 0 & 0 & 0 & 0 & 0 & 0 & 0 & 0 & 0 & 0 & 0 & 0\\
            0 & 1 & 1 & 0 & 0 & 0 & 0 & 0 & 0 & 0 & 0 & 0 & 0 & 0 & 0 & 0 & 0 & 0 & 0 & 0 & 0\\
            0 & 0 & 1 & 2 & 1 & 0 & 0 & 0 & 0 & 0 & 0 & 0 & 0 & 0 & 0 & 0 & 0 & 0 & 0 & 0 & 0\\
            0 & 0 & 0 & 1 & 3 & 3 & 1 & 0 & 0 & 0 & 0 & 0 & 0 & 0 & 0 & 0 & 0 & 0 & 0 & 0 & 0\\
            0 & 0 & 0 & 0 & 1 & 4 & 6 & 4 & 1 & 0 & 0 & 0 & 0 & 0 & 0 & 0 & 0 & 0 & 0 & 0 & 0\\
            0 & 0 & 0 & 0 & 0 & 1 & 5 & 10 & 10 & 5 & 1 & 0 & 0 & 0 & 0 & 0 & 0 & 0 & 0 & 0 & 0\\
            0 & 0 & 0 & 0 & 0 & 0 & 1 & 6 & 15 & 20 & 15 & 6 & 1 & 0 & 0 & 0 & 0 & 0 & 0 & 0 & 0\\
            0 & 0 & 0 & 0 & 0 & 0 & 0 & 1 & 7 & 21 & 35 & 35 & 21 & 7 & 1 & 0 & 0 & 0 & 0 & 0 & 0\\
            0 & 0 & 0 & 0 & 0 & 0 & 0 & 0 & 1 & 8 & 28 & 56 & 70 & 56 & 28 & 8 & 1 & 0 & 0 & 0 & 0\\
            0 & 0 & 0 & 0 & 0 & 0 & 0 & 0 & 0 & 1 & 9 & 36 & 84 & 126 & 126 & 84 & 36 & 9 & 1 & 0 & 0\\
            0 & 0 & 0 & 0 & 0 & 0 & 0 & 0 & 0 & 0 & 1 & 10 & 45 & 120 & 210 & 252 & 210 & 120 & 45 & 10 & 1
            \end{array}\right]  \quad % matrix * vector
            \left[
                \begin{array}{c}
                    f_{n + 2}\\
                    f_{n + 1}\\
                    f_{n}\\
                    f_{n - 1}\\
                    f_{n - 2}\\
                    f_{n - 3}\\
                    f_{n - 4}\\
                    f_{n - 5}\\
                    f_{n - 6}\\
                    f_{n - 7}\\
                    f_{n - 8}\\
                    f_{n - 9}\\
                    f_{n - 10}\\
                    f_{n - 11}\\
                    f_{n - 12}\\
                    f_{n - 13}\\
                    f_{n - 14}\\
                    f_{n - 15}\\
                    f_{n - 16}\\
                    f_{n - 17}\\
                    f_{n - 18}
                    \end{array}\right] \quad = \quad 
            \left[
                \begin{array}{c}
                    f_{n + 2}\\
                    f_{n + 2}\\
                    f_{n + 2}\\
                    f_{n + 2}\\
                    f_{n + 2}\\
                    f_{n + 2}\\
                    f_{n + 2}\\
                    f_{n + 2}\\
                    f_{n + 2}\\
                    f_{n + 2}\\
                    f_{n + 2}
                    \end{array}\right]
        \end{displaymath}

    \caption{Matrix notation of \autoref{triangle:fib:first:order}}
    \label{matrix:notation:triangle:fib:first:order}
\end{sidewaystable}

\subsection{First accumulation}

We provide another generalization produced by \emph{accumulating}
relations in \autoref{triangle:fib:first:order}: in 
\autoref{triangle:fib:first:order:first:accumulation} result of a first
accumulation scan are shown. Leftmost matrix in 
\autoref{matrix:notation:triangle:fib:first:order:first:accumulation}
is known as A$027926$\footnote{\url{http://oeis.org/A027926}} and, 
letting $d_{nk}$ the generic element in row $r$ and column $k$, is
characterized as follows:
\begin{equation}
    \label{eq:triangle:rec:rule:fib}
    d_{00} = 1 \quad d_{n, 2n} = 1 \quad d_{n+1,k+1} = d_{n, k-1} + d_{n, k}  
\end{equation}
therefore we can derive another identity, giving it as a theorem.

\begin{thm}
    \begin{displaymath}
        \left(k + 1\right) f_{n} = \sum_{i=0}^{2 k}  d_{k,2 k - i} f_{n - i}
    \end{displaymath}
\end{thm}

It is interesting to show that every natural number appears in such triangle,
more precisely:
\begin{thm}
    Let $n\in\mathbb{N}$ such that $n > 0$, then:
    \begin{displaymath}
        d_{n,2n-1} = n
    \end{displaymath}
\end{thm}
\begin{proof}
    By induction on $n$.\\
    \begin{itemize}
        \item base case $n=1$ requires to show that $d_{1,1}=1$. \\
            Simply it holds because $d_{1,1} = f_{2} = 1$;
        \item assume by induction hypothesis that $d_{n,2n-1} = n$ holds,
            then prove that $d_{n+1,2(n+1)-1} = n+1$ holds.\\
            By recurrence relation \autoref{eq:triangle:rec:rule:fib} characterizing
            each matrix' coefficient we can rewrite: 
            \begin{displaymath}
                d_{n+1,2(n+1)-1} =d_{n+1,2n+1} = d_{n,2n-1}+d_{n,2n}=n+1
            \end{displaymath}
            since, by construction $d_{n,2n}=1$ and by induction hp $d_{n,2n-1}=n$, as required.
    \end{itemize}
\end{proof}


\begin{sidewaystable}
    \scriptsize
    \begin{eqnarray}
         & f_{n + 2} = f_{n + 2}\\
         & 2 f_{n + 2} = f_{n} + f_{n + 1} + f_{n + 2}\\
         & 3 f_{n + 2} = 2 f_{n} + f_{n - 2} + 2 f_{n - 1} + f_{n + 1} + f_{n + 2}\\
         & 4 f_{n + 2} = 2 f_{n} + f_{n - 4} + 3 f_{n - 3} + 4 f_{n - 2} + 3 f_{n - 1} + f_{n + 1} + f_{n + 2}\\
         & 5 f_{n + 2} = 2 f_{n} + f_{n - 6} + 4 f_{n - 5} + 7 f_{n - 4} + 7 f_{n - 3} + 5 f_{n - 2} + 3 f_{n - 1} + f_{n + 1} + f_{n + 2}\\
         & 6 f_{n + 2} = 2 f_{n} + f_{n - 8} + 5 f_{n - 7} + 11 f_{n - 6} + 14 f_{n - 5} + 12 f_{n - 4} + 8 f_{n - 3} + 5 f_{n - 2} + 3 f_{n - 1} + f_{n + 1} + f_{n + 2}\\
         & 7 f_{n + 2} = 2 f_{n} + f_{n - 10} + 6 f_{n - 9} + 16 f_{n - 8} + 25 f_{n - 7} + 26 f_{n - 6} + 20 f_{n - 5} + 13 f_{n - 4} + 8 f_{n - 3} + 5 f_{n - 2} + 3 f_{n - 1} + f_{n + 1} + f_{n + 2}\\
         & 8 f_{n + 2} = 2 f_{n} + f_{n - 12} + 7 f_{n - 11} + 22 f_{n - 10} + 41 f_{n - 9} + 51 f_{n - 8} + 46 f_{n - 7} + 33 f_{n - 6} + 21 f_{n - 5} + 13 f_{n - 4} + 8 f_{n - 3} + 5 f_{n - 2} + 3 f_{n - 1} + f_{n + 1} + f_{n + 2}\\
         & 9 f_{n + 2} = 2 f_{n} + f_{n - 14} + 8 f_{n - 13} + 29 f_{n - 12} + 63 f_{n - 11} + 92 f_{n - 10} + 97 f_{n - 9} + 79 f_{n - 8} + 54 f_{n - 7} + 34 f_{n - 6} + 21 f_{n - 5} + 13 f_{n - 4} + 8 f_{n - 3} + 5 f_{n - 2} + 3 f_{n - 1} + f_{n + 1} + f_{n + 2}\\
         & 10 f_{n + 2} = 2 f_{n} + f_{n - 16} + 9 f_{n - 15} + 37 f_{n - 14} + 92 f_{n - 13} + 155 f_{n - 12} + 189 f_{n - 11} + 176 f_{n - 10} + 133 f_{n - 9} + 88 f_{n - 8} + 55 f_{n - 7} + 34 f_{n - 6} + 21 f_{n - 5} + 13 f_{n - 4} + 8 f_{n - 3} + 5 f_{n - 2} + 3 f_{n - 1} + f_{n + 1} + f_{n + 2}\\
         & 11 f_{n + 2} = 2 f_{n} + f_{n - 18} + 10 f_{n - 17} + 46 f_{n - 16} + 129 f_{n - 15} + 247 f_{n - 14} + 344 f_{n - 13} + 365 f_{n - 12} + 309 f_{n - 11} + 221 f_{n - 10} + 143 f_{n - 9} + 89 f_{n - 8} + 55 f_{n - 7} + 34 f_{n - 6} + 21 f_{n - 5} + 13 f_{n - 4} + 8 f_{n - 3} + 5 f_{n - 2} + 3 f_{n - 1} + f_{n + 1} + f_{n + 2}
        \end{eqnarray}

    \caption{Relations produced by accumulating equation in 
        \autoref{triangle:fib:first:order}}
    \label{triangle:fib:first:order:first:accumulation}
\end{sidewaystable}

\begin{sidewaystable}
    \begin{displaymath}
        \left[
            \begin{array}{ccccccccccccccccccccc}
            1 & 0 & 0 & 0 & 0 & 0 & 0 & 0 & 0 & 0 & 0 & 0 & 0 & 0 & 0 & 0 & 0 & 0 & 0 & 0 & 0\\
            1 & 1 & 1 & 0 & 0 & 0 & 0 & 0 & 0 & 0 & 0 & 0 & 0 & 0 & 0 & 0 & 0 & 0 & 0 & 0 & 0\\
            1 & 1 & 2 & 2 & 1 & 0 & 0 & 0 & 0 & 0 & 0 & 0 & 0 & 0 & 0 & 0 & 0 & 0 & 0 & 0 & 0\\
            1 & 1 & 2 & 3 & 4 & 3 & 1 & 0 & 0 & 0 & 0 & 0 & 0 & 0 & 0 & 0 & 0 & 0 & 0 & 0 & 0\\
            1 & 1 & 2 & 3 & 5 & 7 & 7 & 4 & 1 & 0 & 0 & 0 & 0 & 0 & 0 & 0 & 0 & 0 & 0 & 0 & 0\\
            1 & 1 & 2 & 3 & 5 & 8 & 12 & 14 & 11 & 5 & 1 & 0 & 0 & 0 & 0 & 0 & 0 & 0 & 0 & 0 & 0\\
            1 & 1 & 2 & 3 & 5 & 8 & 13 & 20 & 26 & 25 & 16 & 6 & 1 & 0 & 0 & 0 & 0 & 0 & 0 & 0 & 0\\
            1 & 1 & 2 & 3 & 5 & 8 & 13 & 21 & 33 & 46 & 51 & 41 & 22 & 7 & 1 & 0 & 0 & 0 & 0 & 0 & 0\\
            1 & 1 & 2 & 3 & 5 & 8 & 13 & 21 & 34 & 54 & 79 & 97 & 92 & 63 & 29 & 8 & 1 & 0 & 0 & 0 & 0\\
            1 & 1 & 2 & 3 & 5 & 8 & 13 & 21 & 34 & 55 & 88 & 133 & 176 & 189 & 155 & 92 & 37 & 9 & 1 & 0 & 0\\
            1 & 1 & 2 & 3 & 5 & 8 & 13 & 21 & 34 & 55 & 89 & 143 & 221 & 309 & 365 & 344 & 247 & 129 & 46 & 10 & 1
            \end{array}
            \right]  \quad % matrix * vector
            \left[
                \begin{array}{c}
                    f_{n + 2}\\
                    f_{n + 1}\\
                    f_{n}\\
                    f_{n - 1}\\
                    f_{n - 2}\\
                    f_{n - 3}\\
                    f_{n - 4}\\
                    f_{n - 5}\\
                    f_{n - 6}\\
                    f_{n - 7}\\
                    f_{n - 8}\\
                    f_{n - 9}\\
                    f_{n - 10}\\
                    f_{n - 11}\\
                    f_{n - 12}\\
                    f_{n - 13}\\
                    f_{n - 14}\\
                    f_{n - 15}\\
                    f_{n - 16}\\
                    f_{n - 17}\\
                    f_{n - 18}
                    \end{array}\right] \quad = \quad 
            \left[
                \begin{array}{c}
                    f_{n + 2}\\
                    2 f_{n + 2}\\
                    3 f_{n + 2}\\
                    4 f_{n + 2}\\
                    5 f_{n + 2}\\
                    6 f_{n + 2}\\
                    7 f_{n + 2}\\
                    8 f_{n + 2}\\
                    9 f_{n + 2}\\
                    10 f_{n + 2}\\
                    11 f_{n + 2}
                    \end{array}\right]
        \end{displaymath}

    \caption{Matrix notation of \autoref{triangle:fib:first:order:first:accumulation}}
    \label{matrix:notation:triangle:fib:first:order:first:accumulation}
\end{sidewaystable}

\begin{sidewaystable}
    \begin{displaymath}
        \left[
            \begin{array}{ccccccccccccccccccccc}
            f_{1} & 0 & 0 & 0 & 0 & 0 & 0 & 0 & 0 & 0 & 0 & 0 & 0 & 0 & 0 & 0 & 0 & 0 & 0 & 0 & 0\\
            f_{1} & f_{2} & f_{1} & 0 & 0 & 0 & 0 & 0 & 0 & 0 & 0 & 0 & 0 & 0 & 0 & 0 & 0 & 0 & 0 & 0 & 0\\
            f_{1} & f_{2} & f_{3} & f_{3} & f_{2} & 0 & 0 & 0 & 0 & 0 & 0 & 0 & 0 & 0 & 0 & 0 & 0 & 0 & 0 & 0 & 0\\
            f_{1} & f_{2} & f_{3} & f_{4} & 4 & f_{4} & f_{2} & 0 & 0 & 0 & 0 & 0 & 0 & 0 & 0 & 0 & 0 & 0 & 0 & 0 & 0\\
            f_{1} & f_{2} & f_{3} & f_{4} & f_{5} & 7 & 7 & 4 & f_{2} & 0 & 0 & 0 & 0 & 0 & 0 & 0 & 0 & 0 & 0 & 0 & 0\\
            f_{1} & f_{2} & f_{3} & f_{4} & f_{5} & f_{6} & 12 & 14 & 11 & f_{5} & f_{2} & 0 & 0 & 0 & 0 & 0 & 0 & 0 & 0 & 0 & 0\\
            f_{1} & f_{2} & f_{3} & f_{4} & f_{5} & f_{6} & f_{7} & 20 & 26 & 25 & 16 & 6 & f_{2} & 0 & 0 & 0 & 0 & 0 & 0 & 0 & 0\\
            f_{1} & f_{2} & f_{3} & f_{4} & f_{5} & f_{6} & f_{7} & f_{8} & 33 & 46 & 51 & 41 & 22 & 7 & f_{2} & 0 & 0 & 0 & 0 & 0 & 0\\
            f_{1} & f_{2} & f_{3} & f_{4} & f_{5} & f_{6} & f_{7} & f_{8} & f_{9} & 54 & 79 & 97 & 92 & 63 & 29 & f_{6} & f_{2} & 0 & 0 & 0 & 0\\
            f_{1} & f_{2} & f_{3} & f_{4} & f_{5} & f_{6} & f_{7} & f_{8} & f_{9} & f_{10} & 88 & 133 & 176 & 189 & 155 & 92 & 37 & 9 & f_{2} & 0 & 0\\
            f_{1} & f_{2} & f_{3} & f_{4} & f_{5} & f_{6} & f_{7} & f_{8} & f_{9} & f_{10} & f_{11} & 143 & 221 & 309 & 365 & 344 & 247 & 129 & 46 & 10 & f_{2}
            \end{array}
            \right]  \quad % matrix * vector
            \left[
                \begin{array}{c}
                    f_{n + 2}\\
                    f_{n + 1}\\
                    f_{n}\\
                    f_{n - 1}\\
                    f_{n - 2}\\
                    f_{n - 3}\\
                    f_{n - 4}\\
                    f_{n - 5}\\
                    f_{n - 6}\\
                    f_{n - 7}\\
                    f_{n - 8}\\
                    f_{n - 9}\\
                    f_{n - 10}\\
                    f_{n - 11}\\
                    f_{n - 12}\\
                    f_{n - 13}\\
                    f_{n - 14}\\
                    f_{n - 15}\\
                    f_{n - 16}\\
                    f_{n - 17}\\
                    f_{n - 18}
                    \end{array}\right] \quad = \quad 
            \left[
                \begin{array}{c}
                    f_{n + 2}\\
                    2 f_{n + 2}\\
                    3 f_{n + 2}\\
                    4 f_{n + 2}\\
                    5 f_{n + 2}\\
                    6 f_{n + 2}\\
                    7 f_{n + 2}\\
                    8 f_{n + 2}\\
                    9 f_{n + 2}\\
                    10 f_{n + 2}\\
                    11 f_{n + 2}
                    \end{array}\right]
        \end{displaymath}

    \caption{Fibonacci substitution within matrix notation of 
        \autoref{matrix:notation:triangle:fib:first:order:first:accumulation}}
    \label{matrix:notation:triangle:fib:first:order:first:accumulation:substituting:fibs}
\end{sidewaystable}

\subsection{Second accumulation}

In \autoref{triangle:fib:first:order:second:accumulation} we report
a second accumulation and in \autoref{matrix:notation:triangle:fib:first:order:second:accumulation}
its matrix notation.

\begin{sidewaystable}
    \scriptsize
    \begin{eqnarray}
        & f_{n + 2} = f_{n + 2}\\
        & 3 f_{n + 2} = f_{n} + f_{n + 1} + 2f_{n + 2}\\
        & 6 f_{n + 2} = 3 f_{n} + f_{n - 2} + 2 f_{n - 1} + 2 f_{n + 1} + 3f_{n + 2}\\
        & 10 f_{n + 2} = 5 f_{n} + f_{n - 4} + 3 f_{n - 3} + 5 f_{n - 2} + 5 f_{n - 1} + 3 f_{n + 1} + 4f_{n + 2}\\
        & 15 f_{n + 2} = 7 f_{n} + f_{n - 6} + 4 f_{n - 5} + 8 f_{n - 4} + 10 f_{n - 3} + 10 f_{n - 2} + 8 f_{n - 1} + 4 f_{n + 1} + 5f_{n + 2}\\
        & 21 f_{n + 2} = 9 f_{n} + f_{n - 8} + 5 f_{n - 7} + 12 f_{n - 6} + 18 f_{n - 5} + 20 f_{n - 4} + 18 f_{n - 3} + 15 f_{n - 2} + 11 f_{n - 1} + 5 f_{n + 1} + 6f_{n + 2}\\
        & 28 f_{n + 2} = 11 f_{n} + f_{n - 10} + 6 f_{n - 9} + 17 f_{n - 8} + 30 f_{n - 7} + 38 f_{n - 6} + 38 f_{n - 5} + 33 f_{n - 4} + 26 f_{n - 3} + 20 f_{n - 2} + 14 f_{n - 1} + 6 f_{n + 1} + 7f_{n + 2}\\
        & 36 f_{n + 2} = 13 f_{n} + f_{n - 12} + 7 f_{n - 11} + 23 f_{n - 10} + 47 f_{n - 9} + 68 f_{n - 8} + 76 f_{n - 7} + 71 f_{n - 6} + 59 f_{n - 5} + 46 f_{n - 4} + 34 f_{n - 3} + 25 f_{n - 2} + 17 f_{n - 1} + 7 f_{n + 1} + 8f_{n + 2}\\
        & 45 f_{n + 2} = 15 f_{n} + f_{n - 14} + 8 f_{n - 13} + 30 f_{n - 12} + 70 f_{n - 11} + 115 f_{n - 10} + 144 f_{n - 9} + 147 f_{n - 8} + 130 f_{n - 7} + 105 f_{n - 6} + 80 f_{n - 5} + 59 f_{n - 4} + 42 f_{n - 3} + 30 f_{n - 2} + 20 f_{n - 1} + 8 f_{n + 1} + 9f_{n + 2}\\
        & 55 f_{n + 2} = 17 f_{n} + f_{n - 16} + 9 f_{n - 15} + 38 f_{n - 14} + 100 f_{n - 13} + 185 f_{n - 12} + 259 f_{n - 11} + 291 f_{n - 10} + 277 f_{n - 9} + 235 f_{n - 8} + 185 f_{n - 7} + 139 f_{n - 6} + 101 f_{n - 5} + 72 f_{n - 4} + 50 f_{n - 3} + 35 f_{n - 2} + 23 f_{n - 1} + 9 f_{n + 1} + 10f_{n + 2}\\
        & 66 f_{n + 2} = 19 f_{n} + f_{n - 18} + 10 f_{n - 17} + 47 f_{n - 16} + 138 f_{n - 15} + 285 f_{n - 14} + 444 f_{n - 13} + 550 f_{n - 12} + 568 f_{n - 11} + 512 f_{n - 10} + 420 f_{n - 9} + 324 f_{n - 8} + 240 f_{n - 7} + 173 f_{n - 6} + 122 f_{n - 5} + 85 f_{n - 4} + 58 f_{n - 3} + 40 f_{n - 2} + 26 f_{n - 1} + 10 f_{n + 1} + 11f_{n + 2}
        \end{eqnarray}

    \caption{Relations produced by accumulating equation in 
        \autoref{triangle:fib:first:order:first:accumulation}}
    \label{triangle:fib:first:order:second:accumulation}
\end{sidewaystable}

\begin{sidewaystable}
    \begin{displaymath}
        \left[
            \begin{array}{ccccccccccccccccccccc}
                1 & 0 & 0 & 0 & 0 & 0 & 0 & 0 & 0 & 0 & 0 & 0 & 0 & 0 & 0 & 0 & 0 & 0 & 0 & 0 & 0\\
                2 & 1 & 1 & 0 & 0 & 0 & 0 & 0 & 0 & 0 & 0 & 0 & 0 & 0 & 0 & 0 & 0 & 0 & 0 & 0 & 0\\
                3 & 2 & 3 & 2 & 1 & 0 & 0 & 0 & 0 & 0 & 0 & 0 & 0 & 0 & 0 & 0 & 0 & 0 & 0 & 0 & 0\\
                4 & 3 & 5 & 5 & 5 & 3 & 1 & 0 & 0 & 0 & 0 & 0 & 0 & 0 & 0 & 0 & 0 & 0 & 0 & 0 & 0\\
                5 & 4 & 7 & 8 & 10 & 10 & 8 & 4 & 1 & 0 & 0 & 0 & 0 & 0 & 0 & 0 & 0 & 0 & 0 & 0 & 0\\
                6 & 5 & 9 & 11 & 15 & 18 & 20 & 18 & 12 & 5 & 1 & 0 & 0 & 0 & 0 & 0 & 0 & 0 & 0 & 0 & 0\\
                7 & 6 & 11 & 14 & 20 & 26 & 33 & 38 & 38 & 30 & 17 & 6 & 1 & 0 & 0 & 0 & 0 & 0 & 0 & 0 & 0\\
                8 & 7 & 13 & 17 & 25 & 34 & 46 & 59 & 71 & 76 & 68 & 47 & 23 & 7 & 1 & 0 & 0 & 0 & 0 & 0 & 0\\
                9 & 8 & 15 & 20 & 30 & 42 & 59 & 80 & 105 & 130 & 147 & 144 & 115 & 70 & 30 & 8 & 1 & 0 & 0 & 0 & 0\\
                10 & 9 & 17 & 23 & 35 & 50 & 72 & 101 & 139 & 185 & 235 & 277 & 291 & 259 & 185 & 100 & 38 & 9 & 1 & 0 & 0\\
                11 & 10 & 19 & 26 & 40 & 58 & 85 & 122 & 173 & 240 & 324 & 420 & 512 & 568 & 550 & 444 & 285 & 138 & 47 & 10 & 1
            \end{array}
            \right]  \quad % matrix * vector
            \left[
                \begin{array}{c}
                    f_{n + 2}\\
                    f_{n + 1}\\
                    f_{n}\\
                    f_{n - 1}\\
                    f_{n - 2}\\
                    f_{n - 3}\\
                    f_{n - 4}\\
                    f_{n - 5}\\
                    f_{n - 6}\\
                    f_{n - 7}\\
                    f_{n - 8}\\
                    f_{n - 9}\\
                    f_{n - 10}\\
                    f_{n - 11}\\
                    f_{n - 12}\\
                    f_{n - 13}\\
                    f_{n - 14}\\
                    f_{n - 15}\\
                    f_{n - 16}\\
                    f_{n - 17}\\
                    f_{n - 18}
                    \end{array}\right] \quad = \quad 
            \left[
                \begin{array}{c}
                    f_{n + 2}\\
                    3 f_{n + 2}\\
                    6 f_{n + 2}\\
                    10 f_{n + 2}\\
                    15 f_{n + 2}\\
                    21 f_{n + 2}\\
                    28 f_{n + 2}\\
                    36 f_{n + 2}\\
                    45 f_{n + 2}\\
                    55 f_{n + 2}\\
                    66 f_{n + 2}
                    \end{array}\right]
        \end{displaymath}

    \caption{Matrix notation of \autoref{triangle:fib:first:order:second:accumulation}}
    \label{matrix:notation:triangle:fib:first:order:second:accumulation}
\end{sidewaystable}


\subsection{Third accumulation}

In \autoref{triangle:fib:first:order:third:accumulation} we report
a third accumulation and in \autoref{matrix:notation:triangle:fib:first:order:third:accumulation}
its matrix notation. Coefficients of $f_{n+2}$ within rightmost vector
form a known sequence $A000292$\footnote{\url{http://oeis.org/A000292}}.

\begin{sidewaystable}
    \scriptsize
    \begin{eqnarray}
        & f_{n + 2} = f_{n + 2}\\
        & 4 f_{n + 2} = f_{n} + f_{n + 1} + 3 f_{n + 2}\\
        & 10 f_{n + 2} = 4 f_{n} + f_{n - 2} + 2 f_{n - 1} + 3 f_{n + 1} + 6 f_{n + 2}\\
        & 20 f_{n + 2} = 9 f_{n} + f_{n - 4} + 3 f_{n - 3} + 6 f_{n - 2} + 7 f_{n - 1} + 6 f_{n + 1} + 10 f_{n + 2}\\
        & 35 f_{n + 2} = 16 f_{n} + f_{n - 6} + 4 f_{n - 5} + 9 f_{n - 4} + 13 f_{n - 3} + 16 f_{n - 2} + 15 f_{n - 1} + 10 f_{n + 1} + 15 f_{n + 2}\\
        & 56 f_{n + 2} = 25 f_{n} + f_{n - 8} + 5 f_{n - 7} + 13 f_{n - 6} + 22 f_{n - 5} + 29 f_{n - 4} + 31 f_{n - 3} + 31 f_{n - 2} + 26 f_{n - 1} + 15 f_{n + 1} + 21 f_{n + 2}\\
        & 84 f_{n + 2} = 36 f_{n} + f_{n - 10} + 6 f_{n - 9} + 18 f_{n - 8} + 35 f_{n - 7} + 51 f_{n - 6} + 60 f_{n - 5} + 62 f_{n - 4} + 57 f_{n - 3} + 51 f_{n - 2} + 40 f_{n - 1} + 21 f_{n + 1} + 28 f_{n + 2}\\
        & 120 f_{n + 2} = 49 f_{n} + f_{n - 12} + 7 f_{n - 11} + 24 f_{n - 10} + 53 f_{n - 9} + 86 f_{n - 8} + 111 f_{n - 7} + 122 f_{n - 6} + 119 f_{n - 5} + 108 f_{n - 4} + 91 f_{n - 3} + 76 f_{n - 2} + 57 f_{n - 1} + 28 f_{n + 1} + 36 f_{n + 2}\\
        & 165 f_{n + 2} = 64 f_{n} + f_{n - 14} + 8 f_{n - 13} + 31 f_{n - 12} + 77 f_{n - 11} + 139 f_{n - 10} + 197 f_{n - 9} + 233 f_{n - 8} + 241 f_{n - 7} + 227 f_{n - 6} + 199 f_{n - 5} + 167 f_{n - 4} + 133 f_{n - 3} + 106 f_{n - 2} + 77 f_{n - 1} + 36 f_{n + 1} + 45 f_{n + 2}\\
        & 220 f_{n + 2} = 81 f_{n} + f_{n - 16} + 9 f_{n - 15} + 39 f_{n - 14} + 108 f_{n - 13} + 216 f_{n - 12} + 336 f_{n - 11} + 430 f_{n - 10} + 474 f_{n - 9} + 468 f_{n - 8} + 426 f_{n - 7} + 366 f_{n - 6} + 300 f_{n - 5} + 239 f_{n - 4} + 183 f_{n - 3} + 141 f_{n - 2} + 100 f_{n - 1} + 45 f_{n + 1} + 55 f_{n + 2}\\
        & 286 f_{n + 2} = 100 f_{n} + f_{n - 18} + 10 f_{n - 17} + 48 f_{n - 16} + 147 f_{n - 15} + 324 f_{n - 14} + 552 f_{n - 13} + 766 f_{n - 12} + 904 f_{n - 11} + 942 f_{n - 10} + 894 f_{n - 9} + 792 f_{n - 8} + 666 f_{n - 7} + 539 f_{n - 6} + 422 f_{n - 5} + 324 f_{n - 4} + 241 f_{n - 3} + 181 f_{n - 2} + 126 f_{n - 1} + 55 f_{n + 1} + 66 f_{n + 2}
        \end{eqnarray}

    \caption{Relations produced by accumulating equation in 
        \autoref{triangle:fib:first:order:second:accumulation}}
    \label{triangle:fib:first:order:third:accumulation}
\end{sidewaystable}

\begin{sidewaystable}
    \begin{displaymath}
        \left[
            \begin{array}{ccccccccccccccccccccc}
                1 & 0 & 0 & 0 & 0 & 0 & 0 & 0 & 0 & 0 & 0 & 0 & 0 & 0 & 0 & 0 & 0 & 0 & 0 & 0 & 0\\
                3 & 1 & 1 & 0 & 0 & 0 & 0 & 0 & 0 & 0 & 0 & 0 & 0 & 0 & 0 & 0 & 0 & 0 & 0 & 0 & 0\\
                6 & 3 & 4 & 2 & 1 & 0 & 0 & 0 & 0 & 0 & 0 & 0 & 0 & 0 & 0 & 0 & 0 & 0 & 0 & 0 & 0\\
                10 & 6 & 9 & 7 & 6 & 3 & 1 & 0 & 0 & 0 & 0 & 0 & 0 & 0 & 0 & 0 & 0 & 0 & 0 & 0 & 0\\
                15 & 10 & 16 & 15 & 16 & 13 & 9 & 4 & 1 & 0 & 0 & 0 & 0 & 0 & 0 & 0 & 0 & 0 & 0 & 0 & 0\\
                21 & 15 & 25 & 26 & 31 & 31 & 29 & 22 & 13 & 5 & 1 & 0 & 0 & 0 & 0 & 0 & 0 & 0 & 0 & 0 & 0\\
                28 & 21 & 36 & 40 & 51 & 57 & 62 & 60 & 51 & 35 & 18 & 6 & 1 & 0 & 0 & 0 & 0 & 0 & 0 & 0 & 0\\
                36 & 28 & 49 & 57 & 76 & 91 & 108 & 119 & 122 & 111 & 86 & 53 & 24 & 7 & 1 & 0 & 0 & 0 & 0 & 0 & 0\\
                45 & 36 & 64 & 77 & 106 & 133 & 167 & 199 & 227 & 241 & 233 & 197 & 139 & 77 & 31 & 8 & 1 & 0 & 0 & 0 & 0\\
                55 & 45 & 81 & 100 & 141 & 183 & 239 & 300 & 366 & 426 & 468 & 474 & 430 & 336 & 216 & 108 & 39 & 9 & 1 & 0 & 0\\
                66 & 55 & 100 & 126 & 181 & 241 & 324 & 422 & 539 & 666 & 792 & 894 & 942 & 904 & 766 & 552 & 324 & 147 & 48 & 10 & 1
            \end{array}
            \right]  \quad % matrix * vector
            \left[
                \begin{array}{c}
                    f_{n + 2}\\
                    f_{n + 1}\\
                    f_{n}\\
                    f_{n - 1}\\
                    f_{n - 2}\\
                    f_{n - 3}\\
                    f_{n - 4}\\
                    f_{n - 5}\\
                    f_{n - 6}\\
                    f_{n - 7}\\
                    f_{n - 8}\\
                    f_{n - 9}\\
                    f_{n - 10}\\
                    f_{n - 11}\\
                    f_{n - 12}\\
                    f_{n - 13}\\
                    f_{n - 14}\\
                    f_{n - 15}\\
                    f_{n - 16}\\
                    f_{n - 17}\\
                    f_{n - 18}
                    \end{array}\right] \quad = \quad 
            \left[
                \begin{array}{c}
                    f_{n + 2}\\
                    4 f_{n + 2}\\
                    10 f_{n + 2}\\
                    20 f_{n + 2}\\
                    35 f_{n + 2}\\
                    56 f_{n + 2}\\
                    84 f_{n + 2}\\
                    120 f_{n + 2}\\
                    165 f_{n + 2}\\
                    220 f_{n + 2}\\
                    286 f_{n + 2}
                    \end{array}\right]
        \end{displaymath}

    \caption{Matrix notation of \autoref{triangle:fib:first:order:third:accumulation}}
    \label{matrix:notation:triangle:fib:first:order:third:accumulation}
\end{sidewaystable}

\subsection{Higher accumulations}

It is possible to continue to accumulate as done in previous
section, producing each time different combinations to which
Fibonacci numbers have to obey. The important fact to keep in mind
is that each produced matrix can be built using \autoref{eq:triangle:rec:rule:fib}.

\section{Base instantiation}

If we instantiate variable $n$ appearing in relations reported in
\autoref{triangle:fib:first:order}, looking for the greater value
that makes a subscript vanish, it is possible to derive the 
new set of relations report in \autoref{triangle:fib:first:order:base:instantiation},
again triangle-shaped.

\begin{equation}
    \hspace{-3cm}
    \begin{array}{c}
        f_{2} = f_{0} + f_{1}\\
        f_{4} = f_{0} + 2 f_{1} + f_{2}\\
        f_{6} = f_{0} + 3 f_{1} + 3 f_{2} + f_{3}\\
        f_{8} = f_{0} + 4 f_{1} + 6 f_{2} + 4 f_{3} + f_{4}\\
        f_{10} = f_{0} + 5 f_{1} + 10 f_{2} + 10 f_{3} + 5 f_{4} + f_{5}\\
        f_{12} = f_{0} + 6 f_{1} + 15 f_{2} + 20 f_{3} + 15 f_{4} + 6 f_{5} + f_{6}\\
        f_{14} = f_{0} + 7 f_{1} + 21 f_{2} + 35 f_{3} + 35 f_{4} + 21 f_{5} + 7 f_{6} + f_{7}\\
        f_{16} = f_{0} + 8 f_{1} + 28 f_{2} + 56 f_{3} + 70 f_{4} + 56 f_{5} + 28 f_{6} + 8 f_{7} + f_{8}\\
        f_{18} = f_{0} + 9 f_{1} + 36 f_{2} + 84 f_{3} + 126 f_{4} + 126 f_{5} + 84 f_{6} + 36 f_{7} + 9 f_{8} + f_{9}\\
        f_{20} = f_{0} + 10 f_{1} + 45 f_{2} + 120 f_{3} + 210 f_{4} + 252 f_{5} + 210 f_{6} + 120 f_{7} + 45 f_{8} + 10 f_{9} + f_{10}\\
        f_{22} = f_{0} + 11 f_{1} + 55 f_{2} + 165 f_{3} + 330 f_{4} + 462 f_{5} + 462 f_{6} + 330 f_{7} + 165 f_{8} + 55 f_{9} + 11 f_{10} + f_{11}\\
        f_{24} = f_{0} + 12 f_{1} + 66 f_{2} + 220 f_{3} + 495 f_{4} + 792 f_{5} + 924 f_{6} + 792 f_{7} + 495 f_{8} + 220 f_{9} + 66 f_{10} + 12 f_{11} + f_{12}\\
        f_{26} = f_{0} + 13 f_{1} + 78 f_{2} + 286 f_{3} + 715 f_{4} + 1287 f_{5} + 1716 f_{6} + 1716 f_{7} + 1287 f_{8} + 715 f_{9} + 286 f_{10} + 78 f_{11} + 13 f_{12} + f_{13}\\
    \end{array}
    \label{triangle:fib:first:order:base:instantiation}
\end{equation}

Using the identity:
\begin{displaymath}
    \sum_{k=1}^{n} f_{2 k} = f_{2 n + 1} - 1
\end{displaymath}
we accumulate relations in \autoref{triangle:fib:first:order:base:instantiation} 
and performing substitution on left hand sides according the cited 
rewriting rule, obtaining:
\begin{equation}
    \begin{array}{c}
        f_{3} - 1 = f_{0} + f_{1}\\
        f_{5} - 1 = 2 f_{0} + 3 f_{1} + f_{2}\\
        f_{7} - 1 = 3 f_{0} + 6 f_{1} + 4 f_{2} + f_{3}\\
        f_{9} - 1 = 4 f_{0} + 10 f_{1} + 10 f_{2} + 5 f_{3} + f_{4}\\
        f_{11} - 1 = 5 f_{0} + 15 f_{1} + 20 f_{2} + 15 f_{3} + 6 f_{4} + f_{5}\\
        f_{13} - 1 = 6 f_{0} + 21 f_{1} + 35 f_{2} + 35 f_{3} + 21 f_{4} + 7 f_{5} + f_{6}\\
        f_{15} - 1 = 7 f_{0} + 28 f_{1} + 56 f_{2} + 70 f_{3} + 56 f_{4} + 28 f_{5} + 8 f_{6} + f_{7}\\
        f_{17} - 1 = 8 f_{0} + 36 f_{1} + 84 f_{2} + 126 f_{3} + 126 f_{4} + 84 f_{5} + 36 f_{6} + 9 f_{7} + f_{8}\\
        f_{19} - 1 = 9 f_{0} + 45 f_{1} + 120 f_{2} + 210 f_{3} + 252 f_{4} + 210 f_{5} + 120 f_{6} + 45 f_{7} + 10 f_{8} + f_{9}\\
        f_{21} - 1 = 10 f_{0} + 55 f_{1} + 165 f_{2} + 330 f_{3} + 462 f_{4} + 462 f_{5} + 330 f_{6} + 165 f_{7} + 55 f_{8} + 11 f_{9} + f_{10}\\
    \end{array}
    \label{triangle:fib:first:order:odds}
\end{equation}
Let us proceed without simplifing using $f_{0}=0$, we will 
do so after. In the following matrix-vector product, let $\mathcal{M}$
be the matrix, so:
\begin{displaymath}
    \left[\begin{array}{ccccccccccc}
        1 & 0 & 0 & 0 & 0 & 0 & 0 & 0 & 0 & 0 & 0\\
        1 & 1 & 0 & 0 & 0 & 0 & 0 & 0 & 0 & 0 & 0\\
        2 & 3 & 1 & 0 & 0 & 0 & 0 & 0 & 0 & 0 & 0\\
        3 & 6 & 4 & 1 & 0 & 0 & 0 & 0 & 0 & 0 & 0\\
        4 & 10 & 10 & 5 & 1 & 0 & 0 & 0 & 0 & 0 & 0\\
        5 & 15 & 20 & 15 & 6 & 1 & 0 & 0 & 0 & 0 & 0\\
        6 & 21 & 35 & 35 & 21 & 7 & 1 & 0 & 0 & 0 & 0\\
        7 & 28 & 56 & 70 & 56 & 28 & 8 & 1 & 0 & 0 & 0\\
        8 & 36 & 84 & 126 & 126 & 84 & 36 & 9 & 1 & 0 & 0\\
        9 & 45 & 120 & 210 & 252 & 210 & 120 & 45 & 10 & 1 & 0\\
        10 & 55 & 165 & 330 & 462 & 462 & 330 & 165 & 55 & 11 & 1
        \end{array}\right]\left[
        \begin{array}{c}
            f_{0}\\
            f_{1}\\
            f_{2}\\
            f_{3}\\
            f_{4}\\
            f_{5}\\
            f_{6}\\
            f_{7}\\
            f_{8}\\
            f_{9}\\
            f_{10}
        \end{array}\right]=\left[
        \begin{array}{c}
            f_{1} - 1\\
            f_{3} - 1\\
            f_{5} - 1\\
            f_{7} - 1\\
            f_{9} - 1\\
            f_{11} - 1\\
            f_{13} - 1\\
            f_{15} - 1\\
            f_{17} - 1\\
            f_{19} - 1\\
            f_{21} - 1
        \end{array}\right]
\end{displaymath}
it is interesting to look for a \emph{binomial transform} of
matrix $\mathcal{M}$, namely find a matrix $\mathcal{T}$ such that
$\mathcal{M} = \mathcal{P}\mathcal{T}$, where $\mathcal{P}$ is 
the Pascal triangle. In order to do that, multiply on the left
both members by $\mathcal{P}^{-1}$, obtaining:
\begin{displaymath}
    \mathcal{T} = \left[\begin{array}{ccccccccccc}
        1 & 0 & 0 & 0 & 0 & 0 & 0 & 0 & 0 & 0 & 0\\
        0 & 1 & 0 & 0 & 0 & 0 & 0 & 0 & 0 & 0 & 0\\
        1 & 1 & 1 & 0 & 0 & 0 & 0 & 0 & 0 & 0 & 0\\
        -1 & 0 & 1 & 1 & 0 & 0 & 0 & 0 & 0 & 0 & 0\\
        1 & 0 & 0 & 1 & 1 & 0 & 0 & 0 & 0 & 0 & 0\\
        -1 & 0 & 0 & 0 & 1 & 1 & 0 & 0 & 0 & 0 & 0\\
        1 & 0 & 0 & 0 & 0 & 1 & 1 & 0 & 0 & 0 & 0\\
        -1 & 0 & 0 & 0 & 0 & 0 & 1 & 1 & 0 & 0 & 0\\
        1 & 0 & 0 & 0 & 0 & 0 & 0 & 1 & 1 & 0 & 0\\
        -1 & 0 & 0 & 0 & 0 & 0 & 0 & 0 & 1 & 1 & 0\\
        1 & 0 & 0 & 0 & 0 & 0 & 0 & 0 & 0 & 1 & 1\\
    \end{array}\right]
\end{displaymath}
matrix $\mathcal{M}$ is known as $A153861$\footnote{\url{http://oeis.org/A153861}},
while matrix $\mathcal{T}$ is known as $A153860$\footnote{\url{http://oeis.org/A153860}}.

    
Apply now the simplification according to $f_{0}=0$, which yield:
\begin{displaymath}
    \begin{array}{c}f_{3} - 1 = f_{1}\\
    f_{5} - 1 = 3 f_{1} + f_{2}\\
    f_{7} - 1 = 6 f_{1} + 4 f_{2} + f_{3}\\
    f_{9} - 1 = 10 f_{1} + 10 f_{2} + 5 f_{3} + f_{4}\\
    f_{11} - 1 = 15 f_{1} + 20 f_{2} + 15 f_{3} + 6 f_{4} + f_{5}\\
    f_{13} - 1 = 21 f_{1} + 35 f_{2} + 35 f_{3} + 21 f_{4} + 7 f_{5} + f_{6}\\
    f_{15} - 1 = 28 f_{1} + 56 f_{2} + 70 f_{3} + 56 f_{4} + 28 f_{5} + 8 f_{6} + f_{7}\\
    f_{17} - 1 = 36 f_{1} + 84 f_{2} + 126 f_{3} + 126 f_{4} + 84 f_{5} + 36 f_{6} + 9 f_{7} + f_{8}\\
    f_{19} - 1 = 45 f_{1} + 120 f_{2} + 210 f_{3} + 252 f_{4} + 210 f_{5} + 120 f_{6} + 45 f_{7} + 10 f_{8} + f_{9}\\
    f_{21} - 1 = 55 f_{1} + 165 f_{2} + 330 f_{3} + 462 f_{4} + 462 f_{5} + 330 f_{6} + 165 f_{7} + 55 f_{8} + 11 f_{9} + f_{10}\\
    \end{array}
\end{displaymath}
in matrix notation:
\begin{displaymath}
    \left[
    \begin{array}{cccccccccc}
        1 & 0 & 0 & 0 & 0 & 0 & 0 & 0 & 0 & 0\\
        3 & 1 & 0 & 0 & 0 & 0 & 0 & 0 & 0 & 0\\
        6 & 4 & 1 & 0 & 0 & 0 & 0 & 0 & 0 & 0\\
        10 & 10 & 5 & 1 & 0 & 0 & 0 & 0 & 0 & 0\\
        15 & 20 & 15 & 6 & 1 & 0 & 0 & 0 & 0 & 0\\
        21 & 35 & 35 & 21 & 7 & 1 & 0 & 0 & 0 & 0\\
        28 & 56 & 70 & 56 & 28 & 8 & 1 & 0 & 0 & 0\\
        36 & 84 & 126 & 126 & 84 & 36 & 9 & 1 & 0 & 0\\
        45 & 120 & 210 & 252 & 210 & 120 & 45 & 10 & 1 & 0\\
        55 & 165 & 330 & 462 & 462 & 330 & 165 & 55 & 11 & 1
        \end{array}\right]  \left[
        \begin{array}{c}
            f_{1}\\
            f_{2}\\
            f_{3}\\
            f_{4}\\
            f_{5}\\
            f_{6}\\
            f_{7}\\
            f_{8}\\
            f_{9}\\
            f_{10}\end{array}\right] = \left[
        \begin{array}{c}
            f_{3} - 1\\
            f_{5} - 1\\
            f_{7} - 1\\
            f_{9} - 1\\
            f_{11} - 1\\
            f_{13} - 1\\
            f_{15} - 1\\
            f_{17} - 1\\
            f_{19} - 1\\
            f_{21} - 1\\
            \end{array}\right]
\end{displaymath}
Infinite matrix is known $A104712$\footnote{\url{http://oeis.org/A104712}}
and can be expressed as a Riordan array:
\begin{displaymath}
    \mathcal{R}\left(\frac{1}{(1-t)^3}, \frac{t}{1-t}\right)
\end{displaymath}

Finally, we can state the following result.
\begin{thm}
    \begin{displaymath}
        f_{2n+1} - 1 = \sum_{k=1}^{n}{{{n+1}\choose{k+1}}f_{k}}
    \end{displaymath}
\end{thm}

\subsection{Another characterization of $f_{2n}$}
Using the identity:
\begin{displaymath}
    \sum_{k=0}^{n} f_{2 k + 1} = f_{2 (n+1)}
\end{displaymath}
we add $1$, which equals to $f_{-1}$ to leave this term
as a symbol, to both members in each equation within 
\autoref{triangle:fib:first:order:odds}, accumulating them
and, finally, applying the above identity on right hand sides, we get:
\begin{equation}
    \begin{array}{c}
        f_{2} = f_{-1} + f_{0}\\
        f_{4} =2 f_{-1} + 2 f_{0} + f_{1}\\
        f_{6} =3 f_{-1} + 4 f_{0} + 4 f_{1} + f_{2}\\
        f_{8} =4 f_{-1} + 7 f_{0} + 10 f_{1} + 5 f_{2} + f_{3}\\
        f_{10} =5 f_{-1} + 11 f_{0} + 20 f_{1} + 15 f_{2} + 6 f_{3} + f_{4}\\
        f_{12} =6 f_{-1} + 16 f_{0} + 35 f_{1} + 35 f_{2} + 21 f_{3} + 7 f_{4} + f_{5}\\
        f_{14} =7 f_{-1} + 22 f_{0} + 56 f_{1} + 70 f_{2} + 56 f_{3} + 28 f_{4} + 8 f_{5} + f_{6}\\
        f_{16} =8 f_{-1} + 29 f_{0} + 84 f_{1} + 126 f_{2} + 126 f_{3} + 84 f_{4} + 36 f_{5} + 9 f_{6} + f_{7}\\
        f_{18} =9 f_{-1} + 37 f_{0} + 120 f_{1} + 210 f_{2} + 252 f_{3} + 210 f_{4} + 120 f_{5} + 45 f_{6} + 10 f_{7} + f_{8}
    \end{array}
    \label{eq:tringle:even:subscripted:fibs:sym}
\end{equation}
moving every term matching $c f_{-1}$, for some $c\in\mathbb{N}$, from right hand sides to
left hand sides again, yields:
\begin{displaymath}
    \begin{array}{c}
        f_{2} - f_{-1}= f_{0}\\
        f_{4} -2 f_{-1} = 2 f_{0} + f_{1}\\
        f_{6} -3 f_{-1} = 4 f_{0} + 4 f_{1} + f_{2}\\
        f_{8} -4 f_{-1} = 7 f_{0} + 10 f_{1} + 5 f_{2} + f_{3}\\
        f_{10} -5 f_{-1} = 11 f_{0} + 20 f_{1} + 15 f_{2} + 6 f_{3} + f_{4}\\
        f_{12} -6 f_{-1} = 16 f_{0} + 35 f_{1} + 35 f_{2} + 21 f_{3} + 7 f_{4} + f_{5}\\
        f_{14} -7 f_{-1} = 22 f_{0} + 56 f_{1} + 70 f_{2} + 56 f_{3} + 28 f_{4} + 8 f_{5} + f_{6}\\
        f_{16} -8 f_{-1} = 29 f_{0} + 84 f_{1} + 126 f_{2} + 126 f_{3} + 84 f_{4} + 36 f_{5} + 9 f_{6} + f_{7}\\
        f_{18} -9 f_{-1} = 37 f_{0} + 120 f_{1} + 210 f_{2} + 252 f_{3} + 210 f_{4} + 120 f_{5} + 45 f_{6} + 10 f_{7} + f_{8}
        \end{array}
\end{displaymath}
Looking at right hand side, we can build a matrix which plays the role
of matrix $\mathcal{M}$ in the above section, namely:
\begin{displaymath}
    \left[\begin{array}{ccccccccc}
        1 & 0 & 0 & 0 & 0 & 0 & 0 & 0 & 0\\
        2 & 1 & 0 & 0 & 0 & 0 & 0 & 0 & 0\\
        4 & 4 & 1 & 0 & 0 & 0 & 0 & 0 & 0\\
        7 & 10 & 5 & 1 & 0 & 0 & 0 & 0 & 0\\
        11 & 20 & 15 & 6 & 1 & 0 & 0 & 0 & 0\\
        16 & 35 & 35 & 21 & 7 & 1 & 0 & 0 & 0\\
        22 & 56 & 70 & 56 & 28 & 8 & 1 & 0 & 0\\
        29 & 84 & 126 & 126 & 84 & 36 & 9 & 1 & 0\\
        37 & 120 & 210 & 252 & 210 & 120 & 45 & 10 & 1
        \end{array}\right]
\end{displaymath}
if we want to find a matrix $\mathcal{T}$ such that 
    $\mathcal{M}=\mathcal{P}\mathcal{T}$, we multiply on the left
    by $\mathcal{P}^{-1}$, obtaining:
\begin{displaymath}
    \mathcal{T} = \left[\begin{array}{ccccccccc}
        1 & 0 & 0 & 0 & 0 & 0 & 0 & 0 & 0\\
        1 & 1 & 0 & 0 & 0 & 0 & 0 & 0 & 0\\
        1 & 2 & 1 & 0 & 0 & 0 & 0 & 0 & 0\\
        0 & 1 & 2 & 1 & 0 & 0 & 0 & 0 & 0\\
        0 & 0 & 1 & 2 & 1 & 0 & 0 & 0 & 0\\
        0 & 0 & 0 & 1 & 2 & 1 & 0 & 0 & 0\\
        0 & 0 & 0 & 0 & 1 & 2 & 1 & 0 & 0\\
        0 & 0 & 0 & 0 & 0 & 1 & 2 & 1 & 0\\
        0 & 0 & 0 & 0 & 0 & 0 & 1 & 2 & 1
        \end{array}\right]
\end{displaymath}
On the other hand, if we simplify according to $f_{0}=0$ and
$f_{-1}=1$, we get:
\begin{displaymath}
    \begin{array}{c}
        f_{2} - 1=0\\
        f_{4} - 2=f_{1}\\
        f_{6} - 3=4f_{1} + f_{2}\\
        f_{8} - 4=10f_{1} + 5 f_{2} + f_{3}\\
        f_{10} - 5=20f_{1} + 15 f_{2} + 6 f_{3} + f_{4}\\
        f_{12} - 6=35f_{1} + 35 f_{2} + 21 f_{3} + 7 f_{4} + f_{5}\\
        f_{14} - 7=56f_{1} + 70 f_{2} + 56 f_{3} + 28 f_{4} + 8 f_{5} + f_{6}\\
        f_{16} - 8=84f_{1} + 126 f_{2} + 126 f_{3} + 84 f_{4} + 36 f_{5} + 9 f_{6} + f_{7}\\
        f_{18} - 9 = 120 f_{1} + 210 f_{2} + 252 f_{3} + 210 f_{4} + 120 f_{5} + 45 f_{6} + 10 f_{7} + f_{8}
        \end{array}
\end{displaymath}
coefficient matrix is known as $A104713$\footnote{\url{http://oeis.org/A104713}},
which is a Riordan array:
\begin{displaymath}
    \mathcal{R}\left(\frac{1}{(1-t)^4}, \frac{t}{1-t}\right)
\end{displaymath}
finally, we can derive the following result.

\begin{thm}
    \begin{displaymath}
        f_{2n} - n = \sum_{k=1}^{n-1}{{{n+1}\choose{k+2}}f_{k}}
    \end{displaymath}
\end{thm}

Moreover, evaluating right hand sides of equations in 
\autoref{eq:tringle:even:subscripted:fibs:sym}, preveenting terms
containing $f_{-1}$ to be substituted, leaving them symbols, we get
the following:
\begin{displaymath}
    \begin{array}{c}
        f_{2} = f_{-1}\\
        f_{4} = 2 f_{-1} + 1\\
        f_{6} = 3 f_{-1} + 5\\
        f_{8} = 4 f_{-1} + 17\\
        f_{10} = 5 f_{-1} + 50\\
        f_{12} = 6 f_{-1} + 138\\
        f_{14} = 7 f_{-1} + 370\\
        f_{16} = 8 f_{-1} + 979\\
        f_{18} = 9 f_{-1} + 2575
    \end{array}
\end{displaymath}
where sequence $(0,1,5,17,50,138,370,979,2575,\ldots)$ is known as 
$A054452$\footnote{\url{http://oeis.org/A054452}}. It is very interesting
to note a link pointed out in the encyclopedia, which shows that
such sequence is the second column of triangle $A054450$, to be 
correct $A054450(2n-1,2)$\footnote{\url{http://oeis.org/A054450}}.

From the encyclopedia again, its generating function $G(t)$ is defined as:
\begin{displaymath}
    G(t) = \frac{t^2}{(1-t)^2 (1-3t+t^2)}
\end{displaymath}
therefore, summing the generating function of natural numbers, namely
$\frac{t}{(1-t)^2}$, we can derive the generating function of even
subscripted Fibonacci numbers, which is:
\begin{displaymath}
    \mathcal{G}\lbrace f_{2n}\rbrace_{n\in\mathbb{N}} = \frac{t}{1 -3t +t^2}
\end{displaymath}

    

\section{Second Order unfolding}

It is possible to use a different recurrence relation to do 
unfolding steps, a natural choice could be to use at each
unfolding step the recently computed one: we call this scheme
\emph{second order} unfolding. Applied to Fibonacci recurrence
relation yield the following equations, each one corresponding
to $0,1,2,3$ unfolding steps:
\begin{displaymath}
    \hspace{-3.5cm}
    \begin{array}{c}
         f_{n} = f_{n - 2} + f_{n - 1}\\
         f_{n} = f_{n - 4} + 2 f_{n - 3} + f_{n - 2}\\
         f_{n} = f_{n - 8} + 4 f_{n - 7} + 5 f_{n - 6} + 2 f_{n - 5} + f_{n - 4} + f_{n - 3}\\
         f_{n} = f_{n - 16} + 8 f_{n - 15} + 26 f_{n - 14} + 44 f_{n - 13} + 42 f_{n - 12} + 25 f_{n - 11} + 13 f_{n - 10} + 7 f_{n - 9} + 3 f_{n - 8} + 3 f_{n - 7} + 3 f_{n - 6} + f_{n - 5}\\
%        f_{n} = f_{n - 32} + 16 f_{n - 31} + 116 f_{n - 30} + 504 f_{n - 29} + 1464 f_{n - 28} + 3010 f_{n - 27} + 4546 f_{n - 26} + 5218 f_{n - 25} + 4755 f_{n - 24} + 3635 f_{n - 23} + 2438 f_{n - 22} + 1473 f_{n - 21} + 863 f_{n - 20} + 541 f_{n - 19} + 333 f_{n - 18} + 177 f_{n - 17} + 88 f_{n - 16} + 49 f_{n - 15} + 37 f_{n - 14} + 34 f_{n - 13} + 28 f_{n - 12} + 17 f_{n - 11} + 8 f_{n - 10} + 4 f_{n - 9} + f_{n - 8}
    \end{array}
\end{displaymath}
previous equations shows other ways to combine Fibonacci numbers
to get other Fibonacci numbers, but combination matrix composed of
such coefficients hasn't a ``closed'' characterization 
(and it is computationally hard to compute).


