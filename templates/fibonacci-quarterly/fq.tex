\documentclass[11pt,reqno]{amsart}
%\setlength{\hoffset}{-.5in}
\setlength{\voffset}{-.25in}
\usepackage{amssymb,latexsym}
\usepackage{graphicx}
\usepackage{url}		%does nice formatting of URLs

\textwidth=6.175in
\textheight=9.0in
\headheight=13pt
\calclayout

\makeatletter
\newcommand{\monthyear}[1]{%
  \def\@monthyear{\uppercase{#1}}}
%   \def\@monthyear{\uppercase{Month Year}}}
\newcommand{\volnumber}[1]{%
  \def\@volnumber{\uppercase{#1}}}
%   \def\@volnumber{\uppercase{Volume , Number }}}
\AtBeginDocument{%
\def\ps@plain{\ps@empty
  \def\@oddfoot{\@monthyear \hfil \thepage}%
  \def\@evenfoot{\thepage \hfil \@volnumber}}
\def\ps@firstpage{\ps@plain}
\def\ps@headings{\ps@empty
  \def\@evenhead{%
    \setTrue{runhead}%
    \def\thanks{\protect\thanks@warning}%
    \uppercase{The Fibonacci Quarterly}\hfil}%
  \def\@oddhead{%
    \setTrue{runhead}%
    \def\thanks{\protect\thanks@warning}%
    \hfill\uppercase{Hypergeometric Template}}%
  \let\@mkboth\markboth
  \def\@evenfoot{%
    \thepage \hfil \@volnumber}%
  \def\@oddfoot{%
    \@monthyear \hfil \thepage}%
  }%
\footskip=25pt
\pagestyle{headings}%
}
\makeatother

\newcommand{\R}{{\mathbb R}}
\newcommand{\Q}{{\mathbb Q}}
\newcommand{\C}{{\mathbb C}}
\newcommand{\N}{{\mathbb N}}
\newcommand{\Z}{{\mathbb Z}}

\theoremstyle{plain}
\numberwithin{equation}{section}
\newtheorem{thm}{Theorem}[section]
\newtheorem{theorem}[thm]{Theorem}
\newtheorem{lemma}[thm]{Lemma}
\newtheorem{example}[thm]{Example}
\newtheorem{definition}[thm]{Definition}
\newtheorem{proposition}[thm]{Proposition}

\newcommand{\helv}{%
  \fontfamily{phv}\fontseries{m}\fontsize{9}{11}\selectfont}

\begin{document}
%% replace the values in the next three lines by the correct information
\monthyear{Month Year}
\volnumber{Volume, Number}
\setcounter{page}{1}

\title{Hypergeometric Template}
\author{Karl Dilcher}
\address{Department of Mathematics and Statistics\\
                Dalhousie University\\
                Halifax, Nova Scotia\\
                B3H 3J5, Canada}
\email{dilcher@mathstat.dal.ca}
\thanks{Research supported in part by the Natural Sciences and Engineering Research Council of Canada and by Emperor Frederick II of Sicily.}
\author{Leonardo Pisano}
\address{Dipartimento di Matematica\\
               Universit\`{a} di Pisa\\
               144 Via Fibonacci\\
               56127 Pisa, Italy}
\email{leo@dm.unipi.it}

\begin{abstract}
This test paper is typeset in the document class \textbf{amsart},
without any additional style files.  Font size and text size are
easy to change; here they are 11pt and $6.175'' \times 9.0''$,
respectively.  This appears to be quite close to the present format
of the \textit{Quarterly}---?.

If we decide to have abstracts for articles in the \textit{Fibonacci
Quarterly}, which I think is a good idea, it will look like this in
this format.
\end{abstract}

\maketitle

\section{Introduction}
This is a shortened and mutilated version of a paper by the first
author, for demonstration purposes only.  As it stands, it makes no
sense; for the real thing, see the \textit{Fibonacci Quarterly}
\textbf{38} (2000), 342--363.

Test of the {\helv font command} to see if it works.

Hypergeometric functions are an important tool in many branches of
pure and applied mathematics, and they encompass most special
functions, including the Chebyshev polynomials.  There are also
well-known connections between Chebyshev polynomials and sequences
of numbers and polynomials related to Fibonacci numbers.  However,
to my knowledge and with one small exception, direct connections
between Fibonacci numbers and hypergeometric functions have not been
established or exploited before.

It is the purpose of this paper to give a brief exposition of
hypergeometric functions, as far as is relevant to the Fibonacci and
allied sequences.  A variety of representations in terms of finite
sums and infinite series involving binomial coefficients are
obtained.  While many of them are well-known, some identities appear
to be new.

The method of hypergeometric functions works just as well for other
sequences, especially the Lucas, Pell, and associated Pell numbers
and polynomials, and also for more general second-order linear
recursion sequences.  However, apart from the final section, we will
restrict our attention to Fibonacci numbers as the most prominent
example of a second-order recurrence.

\section{Hypergeometric Functions}

Almost all of the most common special functions in mathematics and
mathematical physics are particular cases of the \textit{Gauss
hypergeometric series} defined by
\begin{equation}\label{1}
_2 F _1 (a,b; c; z) = \sum_{k=0}^\infty \frac{(a)_k (b)_k}{(c)_k}
\frac{z^k}{k!} ,
\end{equation}
where the \textit{rising factorial} $(a)_k$ is defined by $(a)_0 =
1$ and
\begin{equation}\label{2}
(a)_k = a(a+1) \cdots (a+k-1), \quad (k \ge 1) ,
\end{equation}
for arbitrary $a \in \C$.  The series \eqref{1} is not defined when
$c = -m$, with $m=0, 1, 2, \ldots$, unless $a$ or $b$ are equal to
$-n$, $n = 0, 1, 2, \ldots$, and $n < m$.  It is also easy to see
that the series \eqref{1} reduces to a polynomial of degree $n$ in
$z$ when $a$ or $b$ is equal to $-n$, $n=0, 1, 2, \ldots$.  In all
other cases the series has radius of convergence 1; this follows
from the ratio test and \eqref{2}.  The function defined by the
series \eqref{1} is called the Gauss hypergeometric function.  When
there is no danger of confusion with other types of hypergeometric
series, \eqref{1} is commonly denoted simply by $F(a, b; c; z)$ and
called the hypergeometric series, resp. function.

Most properties of the hypergeometric series can be found in the
well-known reference works \cite{abram}, \cite{magnus} and
\cite{erdelyi} (in increasing order of completeness).  Proofs of
many of the more important properties can be found, e.g., in
\cite{rainville}; see also the important works \cite{bailey} and
\cite{slater}.

At this point we mention only the special case
\begin{equation}\label{3}
F(a, b; b; z) = (1-z)^{-a} ,
\end{equation}
the binomial formula.  The case $a=1$ yields the geometric series;
this gave rise to the term \textit{hypergeometric}.

More properties will be introduced in later sections, as the need
arises.

\section{Fibonacci Numbers}

We will use two different (but related) connections between
Fibonacci numbers and hypergeometric functions.  The first one is
Binet's formula
\begin{equation}\label{4}
F_n = \frac{1}{\sqrt 5} \left[ \left( \frac{1+\sqrt 5}{2} \right) ^n
- \left( \frac{1 - \sqrt 5}{2} \right) ^n \right] ,
\end{equation}
which allows us to use the identity
\begin{equation}\label{5}
F \left( a, \frac{1}{2} + a; \frac{3}{2}; z^2 \right) =
\frac{1}{2z(1-2a)} \left[ (1+z)^{1-2a} - (1-z)^{1-2a} \right]
\end{equation}
(see, e.g., \cite{abram}, (15.1.10)).  If we take $a=(1-n)/2$,
$z=\sqrt 5$, and compare \eqref{5} with \eqref{4}, we obtain
\begin{equation}\label{6}
F_n = \frac{n}{2^{n-1}} F \left( \frac{1-n}{2}, \frac{2-n}{2};
\frac{3}{2}; 5 \right) .
\end{equation}
Note that one of the numbers $(1-n)/2$, $(2-n)/2$ is always a
negative integer (or zero) for $n \ge 1$, so \eqref{6} is in fact a
finite sum and we need not worry about convergence (see, however,
the remark following (4.28)).

Our second approach will be via the well-known connection between
Fibonacci numbers and the Chebyshev polynomials of the second kind,
namely
\begin{equation}\label{7}
F_n = (-i)^{n-1} U_{n-1} \left( \frac{i}{2} \right) .
\end{equation}

\begin{center}
\begin{tabular}{|c|c|c|c|c|c|}
\hline $z$ & $\frac{z}{z-1}$ & $1-z$ & $1-\frac{1}{z}$ &
$\frac{1}{z}$ & $\frac{1}{1-z}$ \\ \hline \hline $5$ & $\frac{5}{4}$
& $-4$ & $\frac{4}{5}$ & $\frac{1}{5}$ & $-\frac{1}{4}$ \\ \hline
$\frac{5}{9}$ & $-\frac{5}{4}$ & $\frac{4}{9}$ & $-\frac{4}{5}$ &
$\frac{9}{5}$ & $\frac{9}{4}$ \\ \hline $\frac{1+\sqrt 5}{2}$ &
$\frac{3+\sqrt 5}{2}$ & $\frac{1-\sqrt 5}{2}$ & $\frac{3-\sqrt
5}{2}$ & $\frac{-1+\sqrt 5}{2}$ & $\frac{-1-\sqrt 5}{2}$ \\ \hline
$\frac{2-\sqrt 5}{4}$ & $9 - 4 \sqrt 5$ & $\frac{2+\sqrt 5}{4}$ & $9
+ 4 \sqrt 5$ & $-8 - 4\sqrt 5$ & $-8 + 4 \sqrt 5$ \\ \hline
\end{tabular}
\end{center}

\begin{center}
TABLE 1. Possible arguments
\end{center}

\section{Linear and Quadratic Transformations}

The next linear transformation formula in the list in \cite{abram},
p.~559, is
\begin{equation}
\begin{split}
F(a, b; c; z) &= \frac{\Gamma (c) \Gamma (c-a-b)}{\Gamma (c-a) \Gamma (c-b)} F(a, b; a+b-c+1; 1-z) \\
&+ (1-z)^{c-a-b} \frac{\Gamma (c) \Gamma (a+b-c)}{\Gamma (a) \Gamma (b)} F(c-a, c-b; c-a-b+1; 1-z). \\
\end{split}
\end{equation}
However, since $a+b-c = -n$ in \eqref{6}, one of the gamma terms in
the numerator is not defined.  Instead, we have to use formula
(15.3.11) in \cite{abram}, p.~559, which in the special case where
$a$ or $b$ is a negative integer and $m$ is a non-negative integer
becomes
\begin{equation}\label{8}
F(a, b; a+b+m; z) = \frac{\Gamma (m) \Gamma (a+b+m)}{\Gamma (a+m)
\Gamma (b+m)} F(a, b; 1-m; 1-z) .
\end{equation}
(For the general case, see, \cite{abram}, (15.3.11), p.~559).  This,
applied to \eqref{6}, gives
\begin{equation}\label{9}
F_n = F \left( \frac{1-n}{2}, \frac{2-n}{2}; 1-n; -4 \right) .
\end{equation}
Here, we have evaluated the gamma terms in (4.7) as follws, using
the duplication formula for $\Gamma (z)$ (see, e.g., \cite{abram},
p.~256):
\[
\begin{split}
\frac{\Gamma (m) \Gamma (a+b+m)}{\Gamma (a+m) \Gamma (b+m)} &= \frac{\Gamma (n)
\Gamma (\frac{3}{2})}{\Gamma (\frac{n}{2} + \frac{1}{2}) \Gamma (\frac{n}{2} + 1)} \\
&= \frac{(2\pi)^{-1/2} 2^{n-1/2} \Gamma (\frac{n}{2}) \Gamma
(\frac{n}{2} + \frac{1}{2}) \frac{1}{2} \sqrt \pi}{\Gamma
(\frac{n}{2} + \frac{1}{2}) \frac{n}{2} \Gamma (\frac{n}{2})} =
\frac{2^{n-1}}{n} .
\end{split}
\]
Another transformation formula similar to (4.6) is
\begin{equation}\label{10}
\begin{split}
F(a, b; c; z) &= \frac{\Gamma (c) \Gamma (b-a)}{\Gamma (b) \Gamma (c-a)} (-z)^{-a} F \left( a, 1-c+a; 1-b+a; \frac{1}{z} \right) \\
&+ \frac{\Gamma (c) \Gamma (a-b)}{\Gamma (a) \Gamma (c-b)} (-z)^{-b}
F \left( b, 1-c+b; 1-a+b; \frac{1}{z} \right) .
\end{split}
\end{equation}

\section{An Irreducibility Result}

This section is taken from another one of the first author's papers,
to illustrate the ``theorem'' and ``proof'' environments.

\begin{proposition}\label{A}
For any pair of nonnegative integers $r$ and $s$, the polynomial
$B_{p-(r+s+1)}^{(r,s)} (x)$ is irreducible for all primes $p>r+s+1$.
\end{proposition}

\begin{proof}
For $r=s=0$, this is a result of Carlitz \cite{carlitz1}.  Let now
$w=r+s+1>1$, and let $d_p^{(r,s)}$ denote the least common multiple
of the denominators of $B_0^{(r,s)}, \ldots , B_{p-w}^{(r,s)}$.  By
the relation \eqref{2} we have
\begin{equation}\label{11}
d_p^{(r,s)} x^{p-w} B_{p-w}^{(r,s)} \left( \frac{1}{x} \right) =
\sum_{k=0}^{p-w} \binom{p-w}{k} d_p^{(r,s)} B_k^{(r,s)} x^k ,
\end{equation}
and clearly this polynomial has integer coefficients.  Now, by
Proposition 5.3 we know that $B_k^{(r,s)}$ is $p$-integral for $0
\le k < p-w$, and that $\alpha _p^{(r,s)} (p-w) = 1$.  Hence $p
\Vert d_p^{(r,s)}$.  Since $p$ does not divide the binomial
coefficients in (6.1), we see that the leading coefficient of the
polynomial is not divisible by $p$, all the other coefficients are
divisible by $p$, but the constant coefficient is not divisible by
$p^2$.  Hence Eisenstein's irreducibility ciriterion applies, and
the polynomial in (6.1), and consequently $B_{p-w}^{(r,s)} (x)$, are
irreducible.
\end{proof}



\begin{thebibliography}{99}

\bibitem{abram}
M. Abramowitz and I. A. Stegun, \emph{Handbook of Mathematical
Functions}, National Bureau of Standards, 1964
\bibitem{agarwal}
A. K. Agarwal, \emph{On a new kind of numbers}, The Fibonacci
Quarterly, \textbf{28.3} (1990), 194--199.
\bibitem{andre}
R. Andr\'{e}-Jeannin, \emph{Generalized complex Fibonacci and Lucas
functions}, The Fibonacci Quarterly, \textbf{29.1} (1991), 13--18.
\bibitem{bailey}
W. N. Bailey, \emph{Generalized Hypergeometric Series}, Cambridge
University Press, Cambridge, 1935.
\bibitem{bruckman}
P. S. Bruckman, \emph{On generating functions with composite
coefficients}, The Fibonacci Quarterly, \textbf{15.3} (1977), 269--275.
\bibitem{carlitz1}
L. Carlitz, \emph{Note on irreducibility of Bernoulli and Euler
polynomials}, Duke Math. J., \textbf{19} (1952), 475--481.
\bibitem{carlitz2}
L. Carlitz, \emph{Some identities of Bruckman}, The Fibonacci
Quarterly, \textbf{13.2} (1975), 121--126.
\bibitem{erdelyi}
A. Erd\'{e}lyi et al., \emph{Higher Transcendental Functions},
Vol.~1, McGraw-Hill Book Company, Inc., New York, 1953.
\bibitem{magnus}
W. Magnus, F. Oberhettinger and R. P. Soni, \emph{Formulas and
Theorems for the Special Functions of Mathematical Physics}, Third
Edition, Springer-Verlag, Berlin, 1966.
\bibitem{rainville}
E. D. Rainville, \emph{Special Functions}, Chelsea Publ. Co., Bronx,
New York, 1971.
\bibitem{slater}
L. J. Slater, \emph{Generalized Hypergeometric Functions}, Cambridge
University Press, Cambridge, 1966.

%The OEIS may be sited either of two ways lised below.
\bibitem{sloane}
OEIS Foundation Inc. (2011), The On-Line Encyclopedia of Integer Sequences, \url{http://oeis.org}.

\bibitem{either or}
OEIS Foundation Inc. (2011), The On-Line Encyclopedia of Integer Sequences, \url{http://oeis.org/A123456}.
\end{thebibliography}

\medskip

\noindent MSC2010: 11B39, 33C05

\end{document}
